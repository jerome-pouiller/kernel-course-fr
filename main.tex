%
% This document is available under the Creative Commons Attribution-ShareAlike
% License; additional terms may apply. See
%   * http://creativecommons.org/licenses/by-sa/3.0/
%   * http://creativecommons.org/licenses/by-sa/3.0/legalcode
%
% Copyright 2010 Jérôme Pouiller <jezz@sysmic.org>
%

% Pour faire une version imprimable, avec les notes sans overlay
% \PassOptionsToClass{notes=show,handout}{beamer}
% Pour en faire un article:
% \documentclass[10pt,ucs,usepdftitle=false,a4paper]{article}
% \usepackage{beamerarticle}

\documentclass[10pt,ucs,usepdftitle=false]{beamer}

% Pour mettre deux pages sur une
% (Préférer l'utilisation d'un post processing)
%\usepackage{pgfpages}
%\pgfpagesuselayout{2 on 1}[a4paper,border shrink=5mm]
%\pgfpagesuselayout{4 on 1}[a4paper,landscape, border shrink=5mm]
%\setbeameroption{show notes on second screen=right}

%
% This document is available under the Creative Commons Attribution-ShareAlike
% License; additional terms may apply. See
%   * http://creativecommons.org/licenses/by-sa/3.0/
%   * http://creativecommons.org/licenses/by-sa/3.0/legalcode
%
% Copyright 2010 Jérôme Pouiller <jezz@sysmic.org>
%
%\documentclass[french,a4paper,10pt]{beamer}
% Configurationde de Beamer
\usetheme{Warsaw}
\usecolortheme[RGB={181,0,0}]{structure}
% Headers et footers
\useoutertheme[subsection=false]{smoothbars}
% Rectangle dans les itemize
\useinnertheme{rectangles}
% Pas de symboles pour la navigation
\setbeamertemplate{navigation symbols}{}
%\setbeamertemplate{background canvas}{%
%  \tikz {%
%    \node at (0,0) {};
%    \node[inner sep=0pt, opacity=0.2] at (0.5\paperwidth,-0.5\paperheight) {\includegraphics[height=27.2mm,width=65.1mm]{logo}};
%  }
%}
% Ajout des numéros de pages
\newcommand*\oldinsertshortitle{}%
\let\oldinsertshorttitle\insertshorttitle%
\renewcommand*\insertshorttitle{\oldinsertshorttitle\hfill\insertframenumber\,/\,\inserttotalframenumber}

\usepackage{ucs}                      % La sortie est en UTF8
\usepackage[utf8x]{inputenc}          % L'entrée aussi
\usepackage[T1]{fontenc}      % Pour la césure des mots accentués
\usepackage{lmodern}          % Pour les lettres francaises
\usepackage[french]{babel}            % Pour la sortie francaise
\usepackage{times}
\usepackage{mathptmx}         % Font (don't forget to install latex-font-recommended)

% This document is available under the Creative Commons Attribution-ShareAlike
% License; additional terms may apply. See
%   * http://creativecommons.org/licenses/by-sa/3.0/
%   * http://creativecommons.org/licenses/by-sa/3.0/legalcode
%
% Copyright 2010 Jérôme Pouiller <jezz@sysmic.org>
%

\usepackage{tikz}
\usetikzlibrary{patterns}
\usetikzlibrary{shapes}
\usetikzlibrary{decorations}
\usetikzlibrary{positioning}
\usetikzlibrary{calc}

\tikzstyle{cgreen}  = [fill=green!20!white,  draw=green!50!black,  line width=1pt, rounded corners=1pt]
\tikzstyle{cred}    = [fill=red!20!white,    draw=red!50!black,    line width=1pt, rounded corners=1pt]
\tikzstyle{cblue}   = [fill=blue!20!white,   draw=blue!50!black,   line width=1pt, rounded corners=1pt]
\tikzstyle{cpurple} = [fill=purple!20!white, draw=purple!50!black, line width=1pt, rounded corners=1pt]
\tikzstyle{corange} = [fill=orange!20!white, draw=orange!50!black, line width=1pt, rounded corners=1pt]
\tikzstyle{ccyan}   = [fill=cyan!20!white,   draw=cyan!50!black,   line width=1pt, rounded corners=1pt]
\tikzstyle{cbrown}  = [fill=brown!20!white,  draw=brown!50!black,  line width=1pt, rounded corners=1pt]
\tikzstyle{cyellow} = [fill=yellow!20!white, draw=yellow!50!black, line width=1pt, rounded corners=1pt]




\ifpdf
  \usepackage{embedfile} 
\else
  \newcommand{\embedfile}[1]{}
\fi

\usepackage[sections,displaymath]{preview}
\PreviewEnvironment{tikzpicture}
\PreviewEnvironment{center}
\PreviewEnvironment{frame}

\usepackage{hyperref} % Doit être chargé avant ntheorem
\newcommand{\file}[1]{\texttt{#1}}
\newcommand{\cmd}[1]{\texttt{#1}}
\renewcommand{\c}[1]{\cmd{#1}}
\newcommand{\email}[1]{\href{mailto:#1}{\nolinkurl{<#1>}}}

% \usepackage[]{version} 
% \usepackage{ntheorem} 
% \setlength{\theorempreskipamount}{0.8ex plus 0.9ex minus 0.1ex}
% \setlength{\theorempostskipamount}{0.8ex plus 0.9ex minus 0.1ex}
% \theoremprework{\vspace{3mm}\hrule}
% \theorempostwork{\hrule\vspace{3mm}}
% \theorembodyfont{\itshape}
% \theoremseparator{.}
% \newtheorem{quest}{Question}[section]

% \theorembodyfont{\normalfont}
% \theoremseparator{}
% \theoremstyle{break}
% \newtheorem*{ans}{Réponse}

% \setlength{\theoremindent}{3mm}
% \theoremseparator{:}
% \theorembodyfont{\itshape}
% \theoremstyle{plain}
% \newtheorem*{man}{Documentation utile}
% \theorembodyfont{\normalfont}
% \newtheorem*{hint}{Remarque}
% \newtheorem*{note}{Note pédagogique}

% \usepackage[vmargin=25mm,hmargin=15mm]{geometry}          % Marges peronnalisées

% Quelques règlage de mise en page
%\renewcommand{\baselinestretch}{1.2} % taille de l'interligne
\setlength{\parindent}{0pt}
\setlength{\parskip}{0.9ex plus 0.5ex minus 0.2ex}

%\usepackage{fancyhdr}        % Fancy page headers
%\lhead{}                     % Top-Left
%\chead{}                     % Top-Center
%\rhead{}                     % Top-Right
%\lfoot{}                     % Bottom-Left
%\cfoot{}                     % Bottom-Center
%\rfoot{}                     % Bottom-Right
%\pagestyle{fancy}

\usepackage{listings}         % Pour mettre en page du code source
\usepackage{color}            % Pour les lien en couleur dans le pdf
\definecolor{colBg}        {rgb}{1,1,0.8}
\definecolor{colKeys}      {rgb}{0,0,1}
\definecolor{colComments}  {rgb}{1,0,0}
\definecolor{colString}    {rgb}{0,0.5,0}
\definecolor{colBasic}     {rgb}{0,0,0}
\definecolor{colIdentifier}{rgb}{0,0,0}
\lstset{%                     % Basic style
  numbers=left,%
  stepnumber=10,%
  numberstyle=\scriptsize,%
%
  basicstyle=\ttfamily\normalsize\color{colBasic},%
  commentstyle=\normalsize\itshape\color{colComments},%
  identifierstyle=\color{colIdentifier},%
  keywordstyle=\bf\ttfamily\color{colKeys},%
  stringstyle=\color{colString},%
  backgroundcolor=\color{colBg},%
%
%  mathescape=true,%
  extendedchars=false,%
%  tabsize=4,%
  columns=flexible,%
  fontadjust=true,%
  frame=lines,%
  showspaces=false,%
  showstringspaces=false,%
%
  emptylines=1,%
  breaklines=true,%
  breakautoindent=true,%     % Inutile avec breaklines=false
  %literate={\'e}{{\'e}}1 {\`e}{{\`e}}1 {\^o}{{\^o}}1
}
\lstset{language=C++}        % ... En C++

\definecolor{darkgreen}{rgb}{0, 0.7, 0}
\definecolor{darkgreen2}{rgb}{0, 0.5, 0}
\definecolor{red2}{rgb}{0.8, 0, 0}
\lstdefinelanguage{diff} {
    morecomment=[f][\color{darkgreen}][0]{+},
    morecomment=[f][\color{red}][0]{-},
    morecomment=[f][\itshape\color{darkgreen2}][0]{+++},
    morecomment=[f][\itshape\color{red2}][0]{---},
    moredelim=[l][\color{cyan}]{\ \@\@},
    moredelim=*[l][\color{blue}]{\@\@},
}

\hypersetup{colorlinks=true,plainpages=false,urlcolor=blue,linkcolor=}





% Apparait sur chaque slide:
%\logo{\pgfimage[height=5mm]{pics/logo}}

\title{Formation au Noyau Linux}
\hypersetup{pdftitle={Formation au Noyau Linux}}
% \subtitle{Sous-titre}
\author[J. Pouiller]{Jérôme Pouiller \email{j.pouiller@sysmic.org}}
\hypersetup{pdfauthor={Jérôme Pouiller}}
\institute[Sysmic]{Sysmic}
%\institute[Sysmic]{\hspace*{1cm}\pgfimage[height=1.5cm]{pics/logo}}
% Plus complet:
% \institute[Sysmic]{
%   \inst{1} \hspace*{1cm}\includegraphics[height=1.5cm]{pics/logo}
%   \and
%   \inst{2} \includegraphics[height=1.5cm]{pics/logo-quiris}
% }
%\date[Juin 2012]{10-12 Juin 2012}
% Pour le PDF seulement:
\subject{Noyau Linux}
\keywords{Linux, OS, sysmic, noyau, kernel}
  
\begin{document}
  
  \begin{frame}[plain]
    \maketitle
    \note[item]{Parler de moi, de mon CV, freelance, sysmic, expertise, Polytech Paris, Tours, Insa Rennes}
    \note[item]{Essaye de travailler sans les slides lorsqu'il y a des commandes}
    \note[item]{Faire un tour de table (autour du café)}
  \end{frame}

  \begin{frame}{Sommaire}
    \begin{itemize} 
      \item Présentation générale
      \item Compiler
      \item Les concepts de développement
      \item Debugguer
      \item L'API
      \item Contribuer
    \end{itemize} 
  \end{frame}
  
  %
% This document is available under the Creative Commons Attribution-ShareAlike
% License; additional terms may apply. See
%   * http://creativecommons.org/licenses/by-sa/3.0/
%   * http://creativecommons.org/licenses/by-sa/3.0/legalcode
%
% Created: 2012-03-13 00:05:28+01:00
% Main authors:
%     - Jérôme Pouiller <jezz@sysmic.org>
%

\part{Présentation}

  De la board
  Linux + LIBC + GNU
  De GNU/Linux
  De Linux
     Monolitique/microkernel
     C vs C++
  Chiffres
  Liste des Architecture
  Versionning 2.4 2.6 3.0

\part{Obtenir le noyau}

  git
    tag: Une version particuliere avec "git co"
    Les differents depots
  http

\part{Obtenir de la documentation}

  Documentation
  API non stable
  generer
     pages de man
     html
  sur le web
     kernel/doc
     lxr.linux.no
  par mail
    connaitre avoir l'auteur avec AUTOR
    connaitre l'auteur avec git blame


  % This document is available under the Creative Commons Attribution-ShareAlike
% License; additional terms may apply. See
%   * http://creativecommons.org/licenses/by-sa/3.0/
%   * http://creativecommons.org/licenses/by-sa/3.0/legalcode
%
% Copyright 2010 Jérôme Pouiller <jezz@sysmic.org>
%

\part{Créer}

\begin{frame}
  \partpage
\end{frame}

\begin{frame}
  \tableofcontents[currentpart]
\end{frame}

\section{Booter par réseau}

\begin{frame}{Booter par réseau}
  \begin{itemize}
  \item Permet de travailler  plus confortablement car evite les cycles
    de scp/flash
  \item Parfois, il faut demarrer  la cible sous Linux pour pouvoir la
    flasher.  C'est donc la seule manière de continuer.
  \item Fonctionne aussi très bien avec des PC
  \item Trois étapes pour démarrer un système
    \begin{enumerate}
    \item Le bootloader configure la carte réseau et place le noyau en
      mémoire
    \item Le noyau s'execute et monte un filesystem réseau
    \item Le premier processus de la vie du système est lancé: \cmd{init}
    \end{enumerate}
    \end{itemize}
\end{frame}

\subsection{Le bootloader}

\begin{frame}[fragile=singleslide]{Le bootloader}{Description}
  \begin{itemize}
  \item    Le    bootloader     se    trouver    souvent    sur    une
    \emph{eeprom}.  Celle-ci   est  directement  mappée   sur  le  bus
    d'adresse
  \item Au minimum, il doit initialiser:
    \begin{itemize}
    \item les timing de la mémoire RAM
    \item les caches CPU
    \end{itemize}
  \end{itemize}
\end{frame}

\begin{frame}[fragile=singleslide]{Le bootloader}{Description}
  \begin{itemize}
  \item Il peut :
    \begin{itemize}
    \item Initialiser une ligne série pour l'utiliser comme terminal
    \item Offrir un prompt et accèder à des options de boot
    \item Initialiser la mémoire flash
    \item Copier le noyau en mémoire
    \item Passer des arguments au noyau
    \item Initialiser le chipset réseau
    \item  Récupérer  des  informations  provenant d'un  serveur  DHCP
      (serveur où récupérer l'image du noyau, indications sur les mise
      à jour disponibles, etc...)
    \item Lire des fichiers provenant du réseau
    \item Lire des fichier par \verb+{X,Y,Z}MODEM+
    \item Ecrire sur la flash
    \item Gérer un système de secours
    \item  Initialiser des  fonctionnalités  cryptographiques (Trusted
      Plateform Manager)
    \end{itemize}
  \end{itemize}
\end{frame}

\begin{frame}{Le bootloader}{Description}
  \begin{itemize}
    \item Il  est très  rare de pouvoir  démarrer Linux  sans bootloader
    fonctionnel
    \item Si votre bootloader  n'est pas fonctionnel, vous aurez souvent
    besoin d'un matériel particulier pour  le mettre à jour (un outils
    capable de flasher l'eeprom)
  \end{itemize}
  Bootloader connus:
  \begin{itemize}
    \item Grub
    \item Syslinux (et son dérivé Isolinux)
    \item U-Boot (utilisé ici)
    \item Redboot
    \item BareBoot (successeur de U-Boot)
  \end{itemize}
\end{frame}

\begin{frame}[fragile=singleslide]{Le bootloader}{Test}
  Testons notre bootloader:
  \begin{itemize}
    \item Démarrez minicom
    \begin{lstlisting}
host$ minicom -D /dev/ttyUSB1
    \end{lstlisting}
  \item Resettez la carte
  \item Appuyez sur une touche pour stopper le démarrage
    \begin{lstlisting}
[...]
Hit any key to stop autoboot:  0
    \end{lstlisting}
  \item Obtenez la liste des commandes
    \begin{lstlisting}
uboot> help
    \end{lstlisting}
  \end{itemize}
  Attention:
  \begin{itemize}
  \item Pas d'historique
  \item Pas de flèches gauche/droite
  \end{itemize}
  En  cas de  problème pour  vous connecter,  vérifiez  vos paramètres
  RS232.
\end{frame}

\subsection{TFTP}

\begin{frame}[fragile=singleslide]{TFTP}{Mise en place}
  \begin{itemize}
  \item Identique au protocole \cmd{ftp} mais plus simple
  \item Permet d'etre implémenté avec très peu de ressource
  \item Mise en place:
    \begin{lstlisting}
host% apt-get install atftp atftpd
host% cp hello/build-arm/hello /srv/tftp
    \end{lstlisting}
  \item Test en local
    \begin{itemize}
    \item Par le shell interactif
      \begin{lstlisting}
host$ atftp 127.0.0.1
> get hello
> exit
      \end{lstlisting}
    \item Par la ligne de commande
      \begin{lstlisting}
host$ atftp 127.0.0.1 -g -r hello
      \end{lstlisting} % $
    \end{itemize}
    \note[item]{Il est possible de parler de inetd et /etc/default/atftpd}
  \item En cas de  problème, consultez les logs (\file{/var/log/syslog}
    ou \file{/var/log/daemon.log})
% A priori, ca n'est plus vrai avec les nouvelles version:
%     \item Remarque: chemin absolu, pas de cd, etc...
%       \begin{lstlisting}
% host$ scp root@target:/.../uImage /var/lib/tftpboot
% host$ atftp 127.0.0.1 -g -l /var/lib/tftpboot/uImage # Test
%       \end{lstlisting}
  \item En  cas d'utilisation intensive (déploiement  sur des milliers
    de  systèmes),  vous   pouvez  utiliser  \cmd{atftpd}  de  manière
    autonome (voir \file{/etc/default/atftpd})
  \item  Il  est  possible  de  modifier le  répertoire  partagé  dans
    \file{/etc/inetd.conf} ou dans \file{/etc/default/atftpd} (suivant
    le mode d'éxecution)
    \note[item]{Parler de inetd}
  \end{itemize}
\end{frame}  

\begin{frame}[fragile=singleslide]{TFTP}
  \note[item]{Il faut fournir le noyau pré-compilé ou le noyau Calao}
  Configuration de la cible pour télécharger et démarrer le noyau. Par RS232:
  \begin{itemize}
  \item Configuration de l'IP
    \begin{lstlisting}
uboot> setenv ipaddr 192.168.1.12
    \end{lstlisting}
  \item Vérification la configuration IP
    \note[item]{U-boot n'est  pas assez évolué pour  répondre aux ping
      (ca nécessite un service qui tourne en arrière plan ce qui n'est
      pas souhaitable pour faire du debug de bas niveau)}
    \begin{lstlisting}
uboot> ping 192.168.1.10
    \end{lstlisting}
  \item Déclaration de notre \emph{host} comme serveur \cmd{tftp}
    \begin{lstlisting}
uboot> setenv serverip 192.168.1.10
uboot> saveenv
    \end{lstlisting}
    \note[item]{Eventuellement,   rebooter  après   le   saveenv  pour
      initialiser correctement le réseau}
  \item Téléchargement du noyau dans une zone libre de la mémoire
    \note[item]{Expliquer d'où provient 21000000}
    \note[item]{L'adresse de  chargement du noyau  dépend de plusieurs
      parmètres: topologie  mémoire, utilisation  de la m'oire  par le
      bootloader, configuration du noyau, taille du noyau, etc..}
%    \note[item]{Reprendre   un    schéma   similaire   à    celui   de
%      \verb+http://www.at91.com/linux4sam/bin/view/Linux4SAM/GettingStarted#Linux4SAM_NandFlash_demo_Memory+
%      mais pour la RAM}
    \begin{lstlisting}
uboot> tftpboot 21000000 uImage
    \end{lstlisting}
  \item Exécution du noyau
    \begin{lstlisting}
uboot> bootm 21000000
    \end{lstlisting}
    \note[item]{Si autostart=yes, pas besoin de cette derniere ligne}
  \item Le noyau  trouve la flash, monte la flash  et charge l'init de
    la flash
    \note[item]{Exo: faites la même chose par ZMODEM}
    \note[item]{Tester avec le NFS éteind}
  \end{itemize}
\end{frame}

\subsection{NFS}

\begin{frame}[fragile=singleslide]{Nfs}
  Comparable au partage réseau de windows.
  \begin{itemize}
  \item Installation
    \begin{lstlisting}
host% apt-get install nfs-kernel-server nfs-common
host$ mkdir nfs-root 
host% vim /etc/exports
    \end{lstlisting} % $
    \note[item]{nfs-common contient le client}
    \note[item]{On peut essayer avec unfs3}
    \note[item]{Autrefois le paquet s'apellait nfs-user-serveur}
  \item Configuration du partage
    \begin{lstlisting}
/home/user/nfs-root 0.0.0.0/0.0.0.0(ro)
    \end{lstlisting}
  \end{itemize}
\end{frame}

\begin{frame}[fragile=singleslide]{Nfs}
  \begin{itemize}
    \item Test sur l'hôte
      \begin{lstlisting}
host$ service nfs-kernel-server restart
host$ mkdir nfs-mount
host% mount -t nfs 127.0.0.1:/home/user/nfs-root nfs-mount
      \end{lstlisting} % $
    \item Vérification des droits root
      \begin{lstlisting}
host$ echo foo > nfs-root/file
host$ chmod 000 nfs-root/file
host$ ls -l nfs-mount
host% cat nfs-mount/file
      \end{lstlisting} % $
      \note[item]{ Le  test sur la cible  ne fonctionne que  si les binaires
        necessaires sont installée}
     \item Test sur la cible (nécessite le support de NFS dans \cmd{mount})
       \begin{lstlisting}
target$ mkdir nfs-mount
target% mount -t nfs 192.168.1.10:/home/user/nfs-root nfs-mount
       \end{lstlisting} % $
     \item En cas de problème, vérifiez les logs: \file{/var/log/daemon.log}
   \end{itemize}
\end{frame}  

%    \note[item] {Ca permet de ne pas leur donner le rootfs. Est-ce qu'on garde?}
%    \note[item] {C'est une très mauvaise idée de dumper une partition montée}
%    \note[item] {Il faut mieux leur faire faire cette manip dans la section sur les MTD}
%     \item Obtenir une copie conforme de la cible
%       \begin{lstlisting}
% target% tar cv / | tftp host -p -l - -r /srv/tftp/root.tar
% host$ gzip /srv/tftp/root.tar
%       \end{lstlisting} % $
\begin{frame}[fragile=singleslide]{Démarrage sur le NFS}
  \begin{itemize}
  \item Modification des arguments passés au noyau
    \note[item]{Testé avec 3e83c0851c2f1483ef92ce845e6f27e3 uImage, du
      site de Calao}
    \note[item]{Testé       avec      50ddb3a654f5e56e3fa1a70216abbb5e
      rootfs.arm.jffs2 du site de Calao}
    \begin{itemize}
    \item Configuration IP
      \begin{lstlisting}
uboot> setenv ipconf ip=192.168.1.13:192.168.1.10:192.168.1.254:255.255.255.0:target:eth0:off
      \end{lstlisting}
    \item Configuration NFS
      \begin{lstlisting}
uboot> setenv nfsconf root=/dev/nfs nfsroot=/home/user/nfs-root
      \end{lstlisting}
    \item La variable \cmd{bootargs} permet de passer des arguments au noyau
      \begin{lstlisting}
uboot> setenv bootargs ${ipconf} ${nfsconf} panic=1
      \end{lstlisting}
    \item Démarrage
      \begin{lstlisting}
uboot> boot
      \end{lstlisting}
    \end{itemize}
  \item Voir \file{Documentation/filesystem/nfs/nfsroot.txt}
  \item Après avoir monté le NFS, le noyau essaye de passer la main au
    programme \cmd{init}
  \end{itemize}
\end{frame}

\section{Compilation des differents éléments}

\subsection{Compilation du noyau}

\begin{frame}[fragile=singleslide]{Récupération des sources}
  Ou récupérer les sources du noyau?
  \begin{enumerate}
  \item Utiliser  les sources souvent fournies.  Il arrive souvent
    qu'elles  contiennent  des  drivers particuliers  et  qu'elles
    soient déjà configurées
  \item Utiliser \cmd{git clone} (nous y reviendrons)
  \item Télecharger sur \file{kernel.org}
  \end{enumerate}
  \note[item]{Fonctionne aussi avec 2.6.37}
  \begin{lstlisting}
host$ wget http://www.kernel.org/pub/linux/kernel/v2.6/linux-2.6.33.7.tar.bz2
host$ tar xvjf linux-2.6.33.7.tar.bz2
  \end{lstlisting}
\end{frame}

\begin{frame}[fragile=singleslide]{Interface de configuration du noyau}
  \begin{itemize}
    \item Utilise Kconfig
      \begin{lstlisting}
host$ make help
host$ mkdir build
host$ make O=build ARCH=arm CROSS_COMPILE=arm-linux- menuconfig
       \end{lstlisting}
     \item  Beaucoup d'options,  mais il  y a  l'aide (\verb+<h>+)  et la
       recherche (\verb+</>+)
    \item La configuration est sauvée dans \file{.config}
    \item    \verb+usb-a9260_defconfig+   permet   de    charger   une
      configuration pré-établie pour notre carte
      \begin{lstlisting}
host$ make O=build ARCH=arm CROSS_COMPILE=arm-linux- usb-a9260_defconfig          
      \end{lstlisting}
    \begin{itemize}
    \item Certains constructeur vous  fournirons un patch ajoutant une
      cible \verb+_defconfig+
    \item D'autre vous fournirons un \file{.config}
    \end{itemize}
  \end{itemize}
\end{frame}

\begin{frame}[fragile=singleslide]{Options à connaitre}
  \begin{itemize}
  \item Kernel Compression (Gzip/LZO)
  \item Support for paging of anonymous memory (swap)
  \item Kernel .config support
  \item Kernel log buffer size
  \item Group CPU scheduler
  \item Initial RAM filesystem and RAM disk
  \item Optimize for size
  \item Configure standard kernel features (for small systems)
  \item EXecute In Place
  \end{itemize}
  \note[item]{TODO P3 faire une liste exaustive}
\end{frame}

\begin{frame}[fragile=singleslide]{Compilation du noyau}
  \begin{itemize}
  \item Verifier les options du type de processeur
  \item Cocher NFS
  \item Le reste ne devrait pas empecher ce demarrer votre cible
  \item La compialtion se lance avec
    \begin{lstlisting}
host$ make XXImage
    \end{lstlisting}
  \item \verb+XX+ fait reference au format de la binaire produite:
    \begin{itemize}
    \item Le premier octet est-il du code?
    \item Respecte-t-il le format ELF?
    \item Y a-t-il un format particulier d'entête à respecter ?
    \end{itemize}
  \item Dans  le doute,  il faut consulter  la documentation  de votre
    bootloader
  \end{itemize}
\end{frame}  

\begin{frame}[fragile=singleslide]{Compilation du noyau}
  Fichiers produits (ou productibles) par la compilation:
  \begin{itemize}
  \item  \verb+vmlinux+:  L'image  ELF  du  noyau.   Lisible  par  les
    debugueurs, certains flasheurs, certain bootloaders
  \item  \verb+vmlinuz+: parfois  équivalent  du \verb+bzImage+,  mais
    normalement, il  s'agit de\verb+vmlinux+ compressé  et strippé des
    informations inutiles  au démarrage. Inutilisable  dans l'état, il
    est nécéssaire de lui adjoindre un bootloader pour le décompresser
    et l'éxecuter.
  \item  \verb+Image+:  \verb+vmlinux+   strippé  et  préfixé  par  un
    mini-bootloader   permettant    de   sauter   sur    la   fonction
    \verb+start_kernel+ de \verb+vmlinux+.
  \item  \verb+bzImage+  et   \verb+zImage+:  \verb+vmlinuz+  avec  le
    bootloader \cmd{bz2} ou \cmd{gz}.
  \item  \verb+xipImage+  :  Idem  \verb+Image+ mais  destiné  à  être
    éxecuté  directement  sur un  \emph{eeprom}  sans  être copier  en
    mémoire au préalable.
  \item  \verb+uImage+:  \verb+Image+ avec  une  entête spéciale  pour
    \emph{u-boot}.
  \end{itemize}
  Il est possible  de générer des image au  format SRecord en utiliant
  \cmd{objcopy}
  \begin{lstlisting}
host$ objcopy -O srec vmlinux vmlinux.srec
   \end{lstlisting}
\end{frame}

\begin{frame}[fragile=singleslide]{Compiler le noyau}
  Dans notre cas, nous utilisons U-Boot (standard)
  \begin{itemize}
  \item Compilation
    \begin{lstlisting}
host% apt-get install uboot-mkimage
host$ make O=build ARCH=arm CROSS_COMPILE=arm-linux- uImage
    \end{lstlisting}
  \item Partage de l'image par TFTP
    \begin{lstlisting}
host% cp build/arch/arm/boot/uImage /srv/tftp/uImage-2.6.33.7
host% ln -s uImage-2.6.33.7 /srv/tftp/uImage
    \end{lstlisting} % $ 
  \item  Au   redémarrage,  le   bootloader  passe  par   un  registre
    l'identifiant  de   la  carte.   Cet   identifiant  (spécifique  à
    l'architecture ARM) est erroné. A  ce stade, il est plus facile de
    corriger   ce   problème   dans   le   noyau   dans   le   fichier
    \file{arch/arm/tools/mach-types}.
    \note[item]{C'est mieux de compiler avec -j3}
    \note[item]{Il faut les laisser démarrer en NFS}
    \note[item]{Il faut  commenter la ligne  1108 et changer  la ligne
      1700}
  \end{itemize}
  \note{Parler de l'installation des modules et de l'option INSTALL\_MOD\_PATH}
\end{frame}


\subsection{Création de l'init}

\begin{frame}[fragile=singleslide]{Démarrage du noyau}
  \begin{itemize}
  \item  A  la fin  du  démarrage  du noyau,  celui  donne  la main  à
    l'éxecutable déclaré  avec \verb+init=+. Par défaut,  il s'agit de
    \file{/sbin/init}
  \item \cmd{init} ne se termine jamais
  \item  Les  arguments  nons  utilisés  par le  noyau  sont  passé  à
    \cmd{init}
  \item On peut  estimer que notre système démarre  à partir du moment
    ou nous  obtenons un shell (c'est  en tous cas  la que la
      plupart des intégrateur Linux embarqué s'arreterons)
  \item Du moins complexe au plus complexe à démarrer:
  \begin{itemize}
    \item \verb+init=/hello-arm-static+
    \item \verb+init=/hello-arm+
    \item \verb+init=/bin/sh+
    \item \verb+init=/sbin/init+
    \end{itemize}
  \end{itemize}
  Effectuons ces tests avec le Rootfs original et un Rootfs vierge.
\end{frame}

\begin{frame}[fragile=singleslide]{Créer l'espace utilisateur}{Créer l'arborescence}
  Nous travaillerons dans un répertoire vierge
  \begin{lstlisting}
host$ mkdir nfs-root-mine
host$ cd nfs-root-mine
host$ ln -s nfs-root-mine nfs-root
  \end{lstlisting} % $
  L'arborescence classique sera:
  \vspace{-2ex}
  \begin{columns}[onlytextwidth,t]
    \begin{column}[t]{0.5\textwidth}
      \begin{itemize}
      \item bin
      \item sbin
      \item usr/bin
      \item usr/sbin
      \item etc
      \end{itemize}
    \end{column}
    \begin{column}[t]{0.5\textwidth}
      \begin{itemize}
      \item dev
      \item proc
      \item sys
      \item tmp
      \item var
      \end{itemize}
    \end{column}
  \end{columns}
  \vspace{2ex}
  Il est possible de ne pas respecter cette arborescence, mais cela
  compliquerait inutilement la chose
\end{frame}

\begin{frame}[fragile=singleslide]{Créer l'espace utilisateur}{Créer l'arborescence}
  Après le démarrage, le noyau ne trouve pas l'init:
    \begin{lstlisting}
[...]
Kernel panic - not syncing: No init found.  Try passing init= option to kernel.
    \end{lstlisting}

    Copions maintenant  \verb+hello-arm-static+ et \verb+hello-arm+ et
    essayons de démarrer avec:
    \begin{lstlisting}
[...]
Cannot open /dev/console: No such file or directory
    \end{lstlisting}
\end{frame}

\begin{frame}[fragile=singleslide]{Démarrage d'une binaire statique}
  Les fichiers devices
  \begin{itemize}
  \item Permettent de communiquer avec le noyau
  \item Il représente plus ou moins chacun un périphérique
  \item Les plus importants sont normés (\file{Documentation/devices.txt})
  \item Il est possible de les créer avec \cmd{mknod}:
    \begin{lstlisting}
host% mknod dev/console c 5 1
    \end{lstlisting}
  \item  Ce  device  est  nécessaire  pour  \cmd{printf}.   Il  serait
    possible  d'écrire un programme  ne nécessitant  aucun accès  à un
    périphérique (exemple: le service réseau echo)
  \end{itemize}
  Nous pouvons maintenant démarrer avec \verb+init=/hello-arm-static+
\end{frame}

\begin{frame}[fragile=singleslide]{Installation des bibliothèques de base}
  Les bibliothèques de base  (\cmd{libc} et apparentés) sont forcement
  fournies avec le cross-compilateur, car elles y sont intimement liés
  \begin{itemize}
  \item Liste des bibliothèques nécessaires
    \begin{lstlisting}
host$ arm-linux-ldd --root . hello-arm
    \end{lstlisting}
    \item Copie
    \begin{lstlisting}
host$ mkdir lib
host$ cp /opt/arm-linux-.../lib/ld-uClibc-0.9.30.2.so lib
host$ cp /opt/arm-linux-.../lib/libuClibc-0.9.30.2.so lib
    \end{lstlisting}
  \end{itemize}
\end{frame}  

\begin{frame}[fragile=singleslide]{Installation des bibliothèques de base}
  \begin{itemize}
  \item Configuration de \cmd{ldconfig}
    \begin{lstlisting}
host$ echo /lib > etc/ld.so.conf
host$ echo /usr/lib >> etc/ld.so.conf
    \end{lstlisting}
  \item Création des liens symboliques
    \begin{lstlisting}
host$ ldconfig -r . -N
    \end{lstlisting}
  \item Création du cache. Le  cache n'est pas obligatoire, mais si il
    existe, il doit être à jour
    \note[item]{Faire un exo sur PC avec ldconfig}
    \begin{lstlisting}
host$ ldconfig -r .
    \end{lstlisting}
  \end{itemize}
  Nous pouvons maintenant démarrer avec \verb+init=/hello-arm+
\end{frame}

\subsection{Compilation de l'espace utilisateur}

\begin{frame}[fragile=singleslide]{Busybox}
  \begin{itemize}
  \item  Contient la plupart  des binaire  nécéssaire pour  démarrer un
    système
  \item Attention, ce ne  sont pas les meme outils que sur  PC. Il y a
    souvent des option non-implémentés ou des comportements différents
  \item Téléchargement
    \begin{lstlisting}
host$ wget http://busybox.net/downloads/busybox-1.18.3.tar.bz2
host$ tar xvjf busybox-1.18.3.tar.bz2
host$ mkdir build
    \end{lstlisting}
  \item On retrouve \cmd{Kconfig}
    \begin{lstlisting}
host$ make O=build CROSS_COMPILE=arm-linux- allnoconfig
host$ make O=build CROSS_COMPILE=arm-linux- menuconfig
    \end{lstlisting}
  \end{itemize}
\end{frame}

\begin{frame}[fragile=singleslide]{Busybox}
  \begin{itemize}
  \item On trouve pleins d'outils
    \note[item]{TODO: mettre en plusieurs colonnes}
  \item    Au   minimum,    cochons    \cmd{ash},   \cmd{init},    les
    \emph{Coreutils},    \cmd{dmesg},   \cmd{ifconfig},   \cmd{mount},
    \cmd{tftp}, \cmd{tar}, \cmd{reboot}, \cmd{vi}
    \note[item]{TODO: mettre en plusieurs colonnes}
  \item         Et         aussi:        \verb+CONFIG_PLATFORM_LINUX+,
    \verb+CONFIG_FEATURE_EDITING+,        \verb+CONFIG_FEATURE_DEVPTS+,
    \verb+CONFIG_LONG_OPTS+,                  \verb+CONFIG_SHOW_USAGE+,
    \verb+FEATURE_VERBOSE_CP_MESSAGE+, \verb+IOCTL_HEX2STR_ERROR+
  \end{itemize}
  \note[item]{TODO P2 Lister de manière exaustive}
\end{frame}

\begin{frame}[fragile=singleslide]{Installation de Busybox}
  \begin{itemize}
  \item Configurons le chemin de destination vers \file{../install}
    \begin{lstlisting}
host$ make O=build CROSS_COMPILE=arm-linux-
host$ make O=build CROSS_COMPILE=arm-linux- install
    \end{lstlisting}
    \note[item]{Compiler avec -j3}
  \item  L'installation créé  des  liens symboliques  vers la  binaire
    \cmd{busybox}
  \item  Sans  Busybox,  toutes  ces  binaires  seraient  séparées  et
    dispersées sur une dizaine de sources
  \item Nous pouvons maintenant démarrer avec \verb+init=/bin/sh+
    \note[item] {Démarrer avec  console=ttyS0,115200 afin d'éviter les
      problème avec le jobs: Non ca ne marche pas}
  \item \verb+init=/bin/init+ pose encore quelques problèmes.
  \end{itemize}
\end{frame}

\begin{frame}[fragile=singleslide]{Compilons init}{Fichiers device}
  Ajoutons  les devices  \file{/dev/tty}  (En fait,  pas  tout à  fait
  obligatoire)
  \begin{lstlisting}
host% mknod dev/ttyS0 c 4 64
host% mknod dev/tty2 c 4 2
host% mknod dev/tty3 c 4 3
host% mknod dev/tty4 c 4 4
  \end{lstlisting}
  \note[item]{tester  avec console=ttyS0,115200  et retirer  cette section : Non, ca ne marche pas}
%   \item \cmd{MAKEDEV} automatise une partie de la création des fichiers
%     devices
%     \note[item]{TODO Voir si on ne  peut pas continuer sans MAKEDEV et
%       l'introduire dans la section "Ajouts"}
%     \begin{lstlisting}
% host% cd dev
% host% MAKEDEV std
% host% MAKEDEV console
%     \end{lstlisting}
  Nous  pouvons  maintenant  démarrer avec  \cmd{init=/bin/init}  mais
  certaines fonctionnalités sont absentes (\cmd{ps}, etc... )
\end{frame}

\begin{frame}[fragile=singleslide]{Configuration de \cmd{init}}
  Il est possible de configurer \cmd{init} avec le fichier
  \file{/etc/inittab}
  \begin{itemize}
  \item Lancement automatique d'un shell
    \begin{lstlisting}
host$ echo "ttyS0::askfirst:/bin/sh" > etc/inittab 
    \end{lstlisting}
  \item Appel d'un script de démarrage.
    \begin{lstlisting}
host$ echo "::sysinit:/etc/init.d/rcS" >> etc/inittab 
host$ mkdir etc/init.d
host$ echo "#!/bin/sh" > etc/init.d/rcS
host$ chmod +x etc/init.d/rcS
    \end{lstlisting}
  \end{itemize}
  Documentation disponible sur la page de man \emph{inittab(5)}
  (disponible ici: \url{http://tfm.cz/man/5/inittab}).\\[2ex]
  Des  fichier de  configuration  de init  et  d'autres utilitaire  de
  busybox sont disponibles dans \file{busybox/examples}
  \note[item]{tester  avec console=ttyS0,115200  et retirer  "ttyS0 de
    l'inittab": Non ca ne marche pas}
\end{frame}

\begin{frame}[fragile=singleslide]{Compilons init}{Fichiers de configuration}
  Il faut monter les partitions \cmd{/proc} et \cmd{/sys}:
  \begin{lstlisting}
target% mount -t proc none /proc 
target% mount -t sysfs none /sys
  \end{lstlisting}
  Automatisation du montage avec \file{inittab}:
  \begin{lstlisting}
host$ echo "::sysinit:mount -t proc none /proc" >> etc/inittab 
host$ echo "::sysinit:mount -t sysfs none /sys" >> etc/inittab 
  \end{lstlisting}
  Nos commandes semblent maintenant correctement fonctionner.
\end{frame}

\begin{frame}[fragile=singleslide]{Autres \cmd{init}}
  Il existe d'autres formes d'\cmd{init}:
  \begin{itemize}
  \item SystemV (celui que nous utilisons)
  \item runit (aussi proposé par Busybox)
  \item upstart (utilisé par Ubuntu)
  \end{itemize}
  Ces \cmd{init},  plus moderne  offre de nouvelle  fonctionnalités et
  plus de robustesse pour le système.
\end{frame}

\begin{frame}{Et si ca ne marche toujours pas?}
  \begin{itemize}
  \item Toujours commencer par  \emph{hello-static}, le moins dépendant
  \item Si il ne fonctionne  pas, recherchez dans le format de binaire
    de  la chaîne  de compilation  et avec  sa compatibilité  avec les
    options du noyaux
  \item  Si \emph{hello-static} fonctionne,  mais pas  \emph{hello} en
    dynamique, cherchez du coté des  du format de la \emph{libc} et de
    sa compatibilité avec le format de binaire \emph{hello}
  \item Si \emph{hello} fonctionne masi  que vous ne pouvez pas lancer
    de shell, cherchez du coté des devices et des droits.
  \item Si le  shell démarre mais pas init, rechechez  du coté des des
    fichiers de configuration et des devices
  \item Vérifier que votre cible  ne change pas d'IP en démarrant (ici
    dans \file{/etc/inittab})
  \item Sinon, cherchez dans les parametres passés au noyau ou dans la
    configuration
  \item Si possible, toujours tester entre chaque modification
  \end{itemize}
\end{frame}

\begin{frame}{Résumé}
  Que contient l'espace utilisateur standard?
  \begin{itemize}
  \item une arborescence
  \item des binaires
  \item des bibliothèques
  \item des fichiers devices
  \end{itemize}
\end{frame}

\section{Quelques ajouts}
 
\subsection{\file{fstab}}

\begin{frame}[fragile=singleslide]{Utilisation de \file{fstab}}
  Il  est   possible  d'automatiser  ce  montage   au  démarrage  avec
  \file{fstab} et \cmd{mount -a}
  \begin{lstlisting}
host$ echo 'none /proc proc'  >> etc/fstab
host$ echo 'none /sys  sysfs' >> etc/fstab
target% mount -a
  \end{lstlisting}

  Nous pouvons utiliser le  fichier \file{etc/inittab} pour monter nos
  partitions automatiquement.
  \begin{lstlisting}
host$ echo "::sysinit:/bin/mount -a" >> etc/inittab 
host$ echo "::shutdown:/bin/mount -r -a" >> etc/inittab 
  \end{lstlisting}
\end{frame}

\subsection{\file{tmpfs}}

\begin{frame}[fragile=singleslide]{Filesystem temporaire}
  Créer un  filesystem en  memoire permet de  protèger notre  flash (à
  durée  de vie  limitée), de  garnatir que  nos système  est toujours
  identique entre chaque démarrage et d'améliorer les performances.
  \begin{itemize}
    \item Création
    \begin{lstlisting}
host$ mkdir tmp
    \end{lstlisting}
  \item Ajout du \emph{stickybit} comme il se doit
    \begin{lstlisting}
host$ chmod +t tmp
    \end{lstlisting}
  \item Montage d'un filesystem contenu en mémoire
    \begin{lstlisting}
target% mount -t tmpfs none tmp
host$ echo 'none /tmp tmpfs' >> etc/fstab
    \end{lstlisting}
  \end{itemize}
\end{frame}

\subsection{\file{MAKEDEV}}

\begin{frame}[fragile=singleslide]{Utilisation de \cmd{MAKEDEV}}
  \cmd{MAKEDEV} permet d'automatiser  la création des fichiers devices
  de base
  \begin{lstlisting}
host$ cd dev
host$ MAKEDEV std
host$ MAKEDEV console
  \end{lstlisting}
\end{frame}

\subsection{\file{ptms}}

\begin{frame}[fragile=singleslide]{Peudo Terminal Multiplexer}
  \cmd{ptmx} (Peudo  Terminal Multiplexer) Permet  de facilement gérer
  l'allocation des terminaux. (Nécessaire pour Dropbear)
  \begin{lstlisting}
host% mknod dev/ptmx c 5 2
host$ mkdir dev/pts
host$ echo 'none /dev/pts devpts' >> etc/fstab
  \end{lstlisting}
\end{frame}

\subsection{\file{mdev}}

\begin{frame}[fragile=singleslide]{Utilisation de \cmd{mdev}}
  \begin{itemize}
  \item Intégré dans Busybox
  \item Uniquement depuis 2.6, nécessite \file{/sys} compilé et monté
  \item Permet de créer les devices à la volée
  \item Sur  les systèmes  très petits et  où l'utilisation  de device
    dynamique n'est  pas nécessaire, onse passe de  \cmd{mdev} à cause
    des dépendances avec le noyau
  \item Création de \file{/dev} sur un disque mémoire
    \begin{lstlisting}
target% mount -t tmpfs none /dev
    \end{lstlisting}
  \item Initialisation \file{/dev} lors du démarrage
    \begin{lstlisting}
target% mdev -s
    \end{lstlisting}
  \item  Installation  de  \cmd{mdev}  comme \emph{handler}  pour  les
    nouveaux périphériques
    \begin{lstlisting}
target% echo /sbin/mdev > /sys/kernel/uevent_helper 
    \end{lstlisting}
%   \item  Automatisation du processus
%     \begin{lstlisting}
% host$ echo 'none /dev tmpfs' >> etc/fstab
% host$ echo "mdev -s" >> etc/rcS
% host$ echo "echo /sbin/mdev > /sys/kernel/uevent_helper" >> etc/rcS
%     \end{lstlisting}
    \note[item]{Brancher une clef USB pour faire la démonstration}
  \end{itemize}
\end{frame}

\subsection{\file{resolv.conf}}

\begin{frame}[fragile=singleslide]{Résolution DNS}
  \begin{itemize}
  \item Ajout de la résolution DNS
    \begin{lstlisting}
host$ echo nameserver 192.168.1.254 > etc/resolv.conf
target% ping www.google.com
    \end{lstlisting}
  \item  Utiliser la  \emph{glibc}  au lieu  de  la \emph{uclibc}.  Il
    serait    alors    nécessaire    de    configurer    le    fichier
    \file{nsswitch.conf}.   Il  serait   possible  de   chosir  parmis
    différente  back-ends  pour  gérer  les  authentifications  et  le
    réseau.
  \end{itemize}
\end{frame}

\subsection{\file{passwd}}

\begin{frame}[fragile=singleslide]{Ajouts d'utilisateurs}
  \begin{itemize}
  \item Ajout d'utilisateurs  (nécessaire pour beaucoup d'applications
    dont Dropbear)
    \begin{lstlisting}
host$ echo 'root:x:0:0:root:/root:/bin/sh' > etc/passwd
host$ echo 'root:x:0:' > etc/group
    \end{lstlisting}
  \item  \verb'root::0:0:root:/root:/bin/sh'  créerait un  utilisateur
    sans mot de passe
    \note[item]{TODO: Parler de shadow, mkpasswd}
  \end{itemize}
\end{frame}

\subsection{Dropbear}

\begin{frame}[fragile=singleslide]{Dropbear}
  \begin{itemize}
  \item Serveur et client ssh
  \item Procédure classique (ou presque)
    \begin{lstlisting}
host$ wget http://matt.ucc.asn.au/dropbear/releases/dropbear-0.53.tar.bz2
host$ tar xvjf dropbear-0.53.tar.bz2
host$ mkdir build
host$ ../configure --disable-zlib --host=arm-linux --build=i386 --prefix=$(pwd)/../install
host$ make PROGRAMS="dropbear dbclient dropbearkey dropbearconvert scp"
host$ make install
    \end{lstlisting}
  \end{itemize}
\end{frame}

\begin{frame}[fragile=singleslide]{Dropbear}
  \begin{itemize}
  \item Gestion des clef authorisées
    \begin{lstlisting}
host$ mkdir -p etc/dropbear
host$ mkdir -p root/.ssh
host$ cp ~/.ssh/authorized_keys root/.ssh
    \end{lstlisting}%$
  \item Génération des clef host:
    \begin{lstlisting}
target% dropbearkey -t rsa -f /etc/dropbear/dropbear_rsa_host_key
target% dropbearkey -t dss -f /etc/dropbear/dropbear_dss_host_key
target% dropbear -E
host$ echo "ttyS0::respawn:/sbin/dropbear -E" >> etc/inittab
host$ ssh root@target
    \end{lstlisting}
  \end{itemize}
  Dropbear  nécessite un  certain  nombre de  fonctionnalité de  notre
  Linux.  Le faire  fonctionner est  un bon  test de  compatibilité de
  notre système
\end{frame}

\section{Bootstrapper la toolchain}

\begin{frame}[fragile=singleslide]{Compiler le cross-compiler et la libc}
  \begin{itemize}
  \item Le compilateur et la \cmd{libc} se compile ensemble
  \item On peut identifier la toolchain à son triplet:
    \begin{itemize}
    \item \verb+<CPU>-<VENDOR>-<SYSTEM>+
    \item \verb+<SYSTEM> ~ <KERNEL>-<OS>+
    \item \verb+<KERNEL> =+ linux
    \item \verb+<OS>+  est une notion  plus floue: gnu,  ulibc, glibc,
      ulibcgnueabi...
    \end{itemize}
  \item  Pour \cmd{gcc},  on abbrège  souvent le  triplet  en omettant
    \verb+<VENDOR>+
  \item Exemples: 
    \begin{itemize}
    \item ppc85-e8541-linux-gnu % A verfieir
    \item arm9-atmel-linux-ulibceabi % A verifier
    \item sh4-st-unknown: Pas de libc, permet de compiler le noyau
      et u-boot, mais pas d'application user
    \item i586-pc-mingw32msvc: Target windows
    \end{itemize}
  \item Attention, ca n'est pas une science exacte
  \end{itemize}
\end{frame}

\begin{frame}[fragile=singleslide]{Compiler le cross-compiler et la libc}
  \begin{itemize}
  \item 3 étapes: 
    \begin{itemize}
    \item On compile \cmd{arm-unknown-gcc}
    \item On configure le noyau
    \item On  compile la  \cmd{libc} avec \cmd{arm-unknown-gcc}  et le
      noyau préconfiguré, on compile le noyau
    \item On compile \cmd{arm-linux-libc-gcc}
    \end{itemize}
  \item Difficultés :
    \begin{itemize}
    \item Assez complexe
    \item Souvent des problèmes de compatibilité entre les versions
    \end{itemize}
  \end{itemize}
\end{frame}    

\begin{frame}[fragile=singleslide]{Compiler le cross-compiler et la libc}
  Différentes \cmd{libc}:
  \begin{itemize}
  \item glibc (``GNU C Library'', la vénérable)
  \item  eglibc  (``Embedded GNU  C  Library'',  meilleur support  des
    diverses  architectures.  Utilisée  depuis récement  sur  diverses
    distributions ``desktop'')
  \item newlib (utilisée par Cygwin) 
  \item uclibc (très utilisées dans l'embarqué)
  \item  dietlibc  (encore  plus   petite  que  ulibc  destinée  à  la
    compilation statique)
  \item bionic (Android)
  \end{itemize}
\end{frame}

\begin{frame}[fragile=singleslide]{Compiler la toolchain}{Crosstool-NG}
  Crosstool-NG :
  \begin{itemize}
  \item Système automatisant toute la procédure et intégrant les patch
    connu pour rendre compatible certains systèmes
  \item Principalement maintenu par Yann Morin (cocoricco)
  \item Il n'existe pas de paquet, nous devons donc le compiler nous même:
    \begin{lstlisting}
host% apt-get install linux-kernel-headers linux-headers texinfo flax yacc ...
host$ wget http://ymorin.is-a-geek.org/download/crosstool-ng/crosstool-ng-1.10.0.tar.bz2
host$ tar xvzf crosstool-ng-1.10.0.tar.bz2
host$ cd crosstool-ng-1.10.0
host$ ./configure
host$ make 
host% make install
    \end{lstlisting}
  \end{itemize}
\end{frame}
   
\begin{frame}[fragile=singleslide]{Compiler la toolchain}{Crosstool-NG}
  \begin{itemize}
  \item Préparation du répertoire de build
    \begin{lstlisting}
$ mkdir ct-build
$ cd ct-build
$ ct-ng help
    \end{lstlisting}
  \item Partons de l'exemple le plus proche de notre configuration
    \begin{lstlisting}
$ ct-ng list-samples
$ ct-ng arm-unknown-linux-uclibcgnueabi
    \end{lstlisting}
  \item Vérifions que le l'exemple compile
    \begin{lstlisting}
$ ct-ng build
    \end{lstlisting}
  \end{itemize}
\end{frame}

\begin{frame}[fragile=singleslide]{Compiler la toolchain}{Crosstool-NG}
  \begin{itemize}
  \item Configurons notre chaine de compilation finale
    \begin{lstlisting}
$ ct-ng clean 
$ ct-ng menuconfig
       \end{lstlisting}
     \item Dans la configuration
       \begin{lstlisting}
Prefix directory: /opt
Number of parallel jobs: 3
# Page de man de gcc + documentation du CPU :
... Emit assembly for CPU: arm926ej-s ...
... Tune for CPU: arm926ej-s ...
# Documentation du CPU + man gcc + wikipedia (http://en.wikipedia.org/wiki/ARM_architecture)
... Architecture level: armv5te ...
... Tuple vendor string : sysmic ...
       \end{lstlisting}
  \end{itemize}
\end{frame}

\begin{frame}[fragile=singleslide]{Compiler la toolchain}{Crosstool-NG}
  \begin{itemize}
     \item Compilons
       \begin{lstlisting}
$ chmod 777 /opt
$ ct-ng build
$ ct-ng tarball
       \end{lstlisting}
     \item Testons en statitique
       \begin{lstlisting}[basicstyle=\ttfamily\scriptsize\color{colBasic}]
host$ /opt/arm-unknown-linux-uclibcgnueabi/bin/arm-unknown-linux-uclibcgnueabi-gcc -static -Wall -I. hello.c -o hello-mygcc-static
target$ ./hello-mygcc-static
       \end{lstlisting}
     \item Testons en dynamique
       \begin{lstlisting}[basicstyle=\ttfamily\scriptsize\color{colBasic}]
host$ /opt/arm-unknown-linux-uclibcgnueabi/bin/arm-unknown-linux-uclibcgnueabi-gcc -Wall -I. hello.c -o hello-mygcc
target$ ./hello-mygcc
	\end{lstlisting}
  \end{itemize}
\end{frame}

\begin{frame}[fragile=singleslide]{Compiler la toolchain}{Crosstool-NG}
  \begin{itemize}
      \item Il  possible (probable) que les bibliothèque  et le format
        de  binaire  ne  soient   pas  compatible  avec  la  toolchain
        existante.    Il  est   alors  nécessaire   de   recopier  les
        bibliothèque  provenant de toolchain  et de  recompiler TOUTES
        les binaires du système
      \item Ajoutons quelques lien symboliques bien pensés
       \begin{lstlisting}[basicstyle=\ttfamily\scriptsize\color{colBasic}]
host$ cd /opt/arm-unknown-linux-uclibcgnueabi/bin
host$ for i in arm-unknown-linux-uclibcgnueabi-*; do 
> ln -s $i arm-linux-${i#arm-unknown-linux-uclibcgnueabi-}; 
> done
	\end{lstlisting}
      \item N'espérez  pas compiler  du premier coup.  Mais autrefois,
        c'était pire!
      \end{itemize}
\end{frame}


\section{Initramfs}

\begin{frame}[fragile=singleslide]{Initramfs}
  \begin{itemize}
  \item Qu'est-ce qu'un initramfs?
  \item Un petit filesystème chargé en mémoire par le bootloader
  \item Successeur de initrd (plus complexe d'utilisation)
  \item Permet déffectuer des actions avant le lancement d'init
  \item Il devrait rester petit et ne s'utiliser que pour continuer le
    boot
  \item  Si  vous  intégrez  busybox  dans  votre  initramfs,  il  est
    préconisé  de  le compiler  en  statique  en  ne gardant  que  les
    composant utiles
  \item  Cas classique:  charger les  modules permettent  d'acceder au
    filesystème:
    \begin{itemize}
    \item M-Sys Disc-On-Chip
    \item Configuration RAID
    \item Configuration réseau pour booter en NFS
    \end{itemize}
  \item Dans notre cas, il n'est pas utile car nous n'avons pas besoin
    de charger de driver particuliers avant le démarrage
  \end{itemize}
\end{frame}  

\begin{frame}[fragile=singleslide]{Initramfs}
  \begin{itemize} 
  \item Création d'un rootfs vraiment minimal:
    \begin{lstlisting}
host$ mkdir initramfs
host$ cd initramfs
host$ mkdir dev
host$ mknod dev/console c 5 1
host% cp .../hello-arm-static init
host$ find . |  cpio -o | gzip > initramfs.gz
    \end{lstlisting}
  \item Comme l'image noyau, l'image initramfs doit être au format u-boot.
    \begin{lstlisting} 
host$ mkimage -A arm -T ramdisk -d initramfs.gz initramfs.gz.img
host$ cp initramfs.gz.img /srv/tftp
uboot> tftp 4000000 initramfs.gz.img
    \end{lstlisting}
  \end{itemize}
\end{frame}

\begin{frame}[fragile=singleslide]{Initramfs}
  \begin{itemize} 
  \item  On  indique  l'adresse   du  ramfs  en  second  paramètre  de
    \cmd{bootm}. Evidement, le noyau doit être configuré pour utiliser
    le ramfs.
    \begin{lstlisting} 
uboot> setenv autostart no
uboot> tftp 22000000 initramfs.gz.img 
uboot> tftp 21000000 uImage 
uboot> bootm 21000000 22000000 
    \end{lstlisting}
  \end{itemize}
  \note[item]{TODO P1 A verifier, tester, completer}
\end{frame}

\section{Flasher le rootfs}

\begin{frame}[fragile=singleslide]{Création du rootfs}
  Pour créer une image \cmd{jffs2}, vous devez au minimum spécifier le
  répertoire  de source,  l'image de  destination, l'endianness  et le
  padding:
  \begin{lstlisting} 
host$ mkfs.jffs2 -l -p -r nfs-root -o rootfs.jffs2
  \end{lstlisting}
  Afin de  rendre notre  image plus perfomante,  tunons la  taille des
  pages,  supprimons  les  cleanmarkers  (on pourrait  aussi  utiliser
  \verb+-c 0+) et changeons le propriétaires des fichier en root:
  \begin{lstlisting} 
host$ mkfs.jffs2 -l -p -q -n -s 2048 -e 128 -r nfs-root -o rootfs.jffs2
  \end{lstlisting}
\end{frame}

\begin{frame}[fragile=singleslide]{Recopier le système sur la flash}
  Le  plus  générique pour  recopier  la flash  sur  la  cible est  de
  démarrer Linux par un autre moyen  (NFS ou partition de rescue) et d
  utiliser les outils Linux.
  \begin{itemize} 
  \item Toujours commencer par effacer le contenu précédent (on risque
    sinon des corruptions sur le filesystem)
    \begin{lstlisting}
target% flash_eraseall /dev/mtd1
target% cd /tmp
    \end{lstlisting}
  \end{itemize}
\end{frame}

\begin{frame}[fragile=singleslide]{Recopier le système sur la flash}
  \begin{itemize} 
  \item Téléchargeons  le rootfs sur  notre cible. Nous  pouvons avoir
    des problèmes  de place pour stoker  cette image.  Il  est dans ce
    cas préférable de  travailler sur NFS (mais ca  n'est pas toujours
    possible).
    \begin{lstlisting} 
target% tftp -g -r rootfs.jffs2
target% flashcp rootfs.jffs2 /dev/mtd1
    \end{lstlisting}
  \item Test. Les devices \verb+mtd*+ et \verb+mtdblock*+ doivent être présents.
    \begin{lstlisting}
target% mkdir /tmp/jffs2
target% mount -t jffs2 /dev/mtdblock1 /tmp/jffs2
    \end{lstlisting} 
  \end{itemize}
\end{frame}

\begin{frame}[fragile=singleslide]{Recopier le système sur la flash}
  Il est possible d'effectuer cette opération à partir de U-Boot.
  \begin{itemize} 
  \item Nous allons utiliser TFTP sans démarrer dessous. Désactivons l'autostart
    \begin{lstlisting} 
uboot> set autostart no
    \end{lstlisting} 
  \item Toujours effacer la flash avant de flasher
    \begin{lstlisting} 
uboot> nand erase clean 1000000 8800000
    \end{lstlisting} 
  \item On place l'image en mémoire afin de la flasher
    \begin{lstlisting} 
uboot> tftp 21000000 rootfs.arm.jffs2
uboot> nand write 21000000 1000000 5A0000
    \end{lstlisting} 
  \end{itemize}
\end{frame}

\begin{frame}[fragile=singleslide]{Recopier le noyau sur la flash}
  L'opération est similaire au filesystem. Dans la plupart des cas, on
  place le  noyau à  part sur  une autre partition  de la  flash. Nous
  n'avons alors pas besoin de filesystem:
  \begin{itemize} 
  \item Sous Linux
    \begin{lstlisting}
target% flash_eraseall /dev/mtd0
target% cd /tmp
target% tftp -g -r uImage
target% flashcp uImage /dev/mtd0
    \end{lstlisting}
  \item Sous U-boot
    \begin{lstlisting} 
uboot> tftp 21000000 uImage
uboot> nand erase clean 400000 200000
uboot> nand write 21000000 400000 200000
    \end{lstlisting} 
  \end{itemize}
\end{frame}

\begin{frame}[fragile=singleslide]{Automatisation}
  Les  options  non utilisées  par  le  noyau  sont passée  au  script
  d'initialisation.   Ajoutons  une   option  permettant   de  flasher
  automatiquement notre cible:
  \begin{lstlisting}
if [ $1 == FLASH ]
  flash_eraseall /dev/mtd0
  flash_eraseall /dev/mtd1
  cd /tmp
  flashcp uImage /dev/mtd0
  flashcp rootfs.jffs2 /dev/mtd1
fi
  \end{lstlisting}
\end{frame}

\begin{frame}[fragile=singleslide]{Monter le rootfs sur l'hôte}
  L'hôte  ne possèdant  pas  de  flash, nous  devons  la simuler  pour
  pouvoir monter le rootfs:
  \begin{itemize} 
  \item  Chargeons le  module  permettant l'émulation  d'une flash  et
    copions notre image sur la flash virtuelle
    \begin{lstlisting}
host% modprobe mtdram total_size=65536 erase_size=256
host% flashcp rootfs.jffs2 /dev/mtd0
    \end{lstlisting}
  \item  Alternativement, il  est  possible de  directement mapper  un
    fichier sur une flash avec \file{block2mtd}
    \begin{lstlisting}
host% losetup /dev/loop0 rootfs.jffs2
host% modprobe block2mtd block2mtd=/dev/loop0
    \end{lstlisting} 
  \end{itemize}
\end{frame}

\begin{frame}[fragile=singleslide]{Monter le rootfs sur l'hôte}
  \begin{itemize} 
  \item Chargeons la couche d''emulation par block et le filesystem
    \begin{lstlisting}
host% modprobe mtdblock
host% modprobe jffs2
    \end{lstlisting}
  \item Copie de notre image
    \begin{lstlisting} 
host% mkdir jffs2-mount
host% mount -t jffs2 /dev/mtdblock0 jffs2-mount
    \end{lstlisting} 
  \end{itemize}
  % http://wiki.maemo.org/Modifying_the_root_image
\end{frame}

\note[item]{Est-ce que l'on ne devrait pas placer les chapitre sur u-boot et SAm-BA ici? Je pense que oui}

\section{Simuler}

\begin{frame}[fragile=singleslide]{Qu'est-ce que Qemu?}
  \begin{itemize}
  \item Machine Virtuelle
  \item Comparé à VirtualBox et VMWare:
    \begin{itemize}
    \item Plus polyvalent
    \item ...mais un peu  moins intuitif (possibilite d'utiliser qtemu
      ou qemulator)
    \end{itemize}
  \item Rapide car:
    \begin{itemize}
    \item Utilise la compilation JIT (Just-In-Time)
    \item Utilise des extention du processeur pour gérer les
      addresses virtuelle (Module KVM)
      \begin{lstlisting}
host% apt-get install qemu-kvm-extras
      \end{lstlisting}
    \end{itemize}
  \end{itemize}
\end{frame}

\begin{frame}[fragile=singleslide]{Qu'est-ce que Qemu?}
  Emule:
  \begin{itemize}
  \item Simplement un jeux d'instruction 
    \begin{itemize}
    \item Toutes les grandes architectures sont supportée
    \item  Les  appels  système  sont  alors bindéz  vers  les  appels
      systèmes de l'hote
      \begin{lstlisting}
host% qemu-arm ./hello-arm-static
host% qemu-arm -L ../arm-linux-uclibceabi/ ./hello-arm-debug
      \end{lstlisting}
    \item Utilisé par Scratchbox
      \begin{itemize}
      \item Scratchbox crée un chroot et utiliser fakechroot
      \item Qemu  doit être compilé  en static pour être  utilisé avec
        fakechroot (sombre histoire de libld) % A vérifier
      \end{itemize}
    \item Ne permet pas d'avoir un périphérique virtuel
    \end{itemize}
  \item Un système
    \begin{itemize}
    \item Il est possible  d'émuler des periphérique non existants sur
      PC
    \item  Il  est  possible  avec  un peu  d'effort  de  simuler  des
      périphériques spéciaux.
    \item Simulation de système complets. QA
    \end{itemize}
  \end{itemize}
\end{frame}

\begin{frame}[fragile=singleslide]{Qu'est-ce que Qemu?}
  \begin{itemize}
  \item Le  port série de l'AT91  n'est pas présent dans  la liste des
    périphériques de Qemu
  \item On va donc simuler une autre board
  \end{itemize}
  \begin{lstlisting}[basicstyle=\ttfamily\scriptsize\color{colBasic}]
host$ wget http://wiki.qemu.org/download/arm-test-0.2.tar.gz
host$ tar xvzf arm-test-0.2.tar.gz
host$ qemu-system-arm -M integrator -cpu arm926 -m 16 -kernel zImage.integrator -initrd arm_root.img
host$ qemu-system-arm -M integrator -cpu arm926 -m 16 -kernel zImage.integrator -initrd arm_root.img -nographic -append "console=ttyAMA0"
  \end{lstlisting}
\end{frame}

% Acceder aux disque Qemu sans  démarrer Qemu: Before you can use this
% command you  need to  have the NBD  kernel module  loaded ('modprobe
% nbd') then you  can map the virtual harddisk file  to the nbd device
% ('qemu-nbd -vc/dev/nbd0  HD image.qcow') this will work  with any HD
% image format supported by qemu.   This will create a device for each
% partition in the virtual hard disk  file. You can then mount the the
% partions  like 'sudo mount  /dev/nbd0p1 /mnt'  When done  you should
% unmount ('umount  /mnt') and  disconnect the HD  image from  the nbd
% device ('qemu-nbd -d /dev/nbd0') -> On peut aussi utiliser xmount

% Dimensionner une syst`eme Linux embarqué
%    \item PC ou autre?
%    \item Linux ou autre?
%    \item Voir livre de Pierre Ficheux

% TODO: Compiler X et Qt pour Integrator

\section{Automatisation}

\begin{frame}[fragile=singleslide]{Automatisation}{Buildroot}
  \begin{itemize}
  \item But: créer un filesystem root
  \item Utilisation de Kconfig
  \item Thomas Petazzoni
  \item Permet d'automatiser la création  de la toolchain, du noyau,
    de busybox et d'environ 300 outils
    \begin{itemize}
    \item serveurs http, ftp, ssh, etc..
    \item outils réseau, wireless, bluetooth, etc...
    \item Serveur X, gtk
    \end{itemize}
  \item Architecture assez propre
  \item Extention relativement simple ou nous retrouvons les commandes
    utilisée pour compiler des programmes tierces
  \item C'est un peu l'extention de Busybox
  \end{itemize}
\end{frame}

\begin{frame}[fragile=singleslide]{Automatisation}{Buildroot}
  \begin{itemize}
  \item Récupération des sources
    \begin{lstlisting}
host$ wget http://buildroot.uclibc.org/downloads/buildroot-2010.11.tar.bz2
host$ tar xvf buildroot-2010.11.tar.bz2
host$ cd buildroot-2010.11
    \end{lstlisting} 
  \item  Utilisation  d'une configuraiton  pré-établie  comme base  de
    configuration
    \begin{lstlisting}
host$ make usb-a9260_defconfig
host$ make menuconfig
host$ make linux26-menuconfig
host$ make uclibc-menuconfig
host$ make busybox-menuconfig
host$ make all
    \end{lstlisting}
  \item Documentation: \url{http://buildroot.uclibc.org/buildroot.html}
  \end{itemize}
\end{frame}

\begin{frame}[fragile=singleslide]{Automatisation}{OpenEmbedded}
  \begin{itemize}
  \item But: créer une distribution type Debian
  \item Gère un système de paquets
  \item Assez lourd à la configuration
  \item Tres lourd de créer une nouvelle cible
  \item Beaucoup de paquets sont deja préparés (~1800)
    \begin{itemize}
    \item .. Principalement, toute la suite Gtk Opie
    \end{itemize}
%  \item remplacé par Buildroot et Scratchbox 
%   \note[item]{Tout ici: \url{http://www.at91.com/linux4sam/bin/view/Linux4SAM/OpenEmbeddedAngstromBuild}}
%   \note[item]{Résultat ici: \url{ftp://www.at91.com/pub/demo/linux4sam_2.0/linux4sam-angstrom-at91sam9260ek.zip}}
%   \item  Récupératon d'une version d'OpenEmbedded ppackagə pour notre architecture
%   \begin{lstlisting}
% wget ftp://ftp.linux4sam.org/pub/oe/linux4sam_x.y/oe_at91sam.tgz
% tar xvzf oe_at91sam.tgz

% vim oe_at91sam/conf/local.conf
% source ./oe_env.sh
% apt-get install bitbake
% bitbake base-image
% bitbake console-at91sam9-image
% bitbake x11-at91sam9-image
%   \end{lstlisting}
    \note[item]{TODO P4: Faire fonctionne OpenEmbedded}
  \end{itemize}
\end{frame}

\begin{frame}[fragile=singleslide]{Automatisation}{Strachbox}
  \begin{itemize}
  \item but: créer une distribution type Debian
  \item Utilisé  par Meego (Nokia,  intel, Renault, etc..)  et Android
    (en cours)
  \item packagé: \verb+apt-get install scratchbox2+
  \item Coquille vide
    \begin{itemize}
    \item On peut installer l'environement de Maemo ou de Android
    \end{itemize}
  \item Basé sur un fonctionnement hybrid avec qemu
    \begin{itemize}
    \item On compile avec le  cross-compiler mais le reste est executé
      avec qemu ou en ssh sur la cible
    \item S'utilie comme un environement de compilation normal
    \item Rend la cross-compilation transparente
    \item Fonctionne pour ARM and  x86 targets (PowerPC, MIPS and CRIS
      targets are experimental)
    \item  Permet de  conpiler certains  packets ne  peuvent  pas etre
      cross-compilé (Notablement Python)
    \item Depend du bon fonctionnement de qemu
    \end{itemize}
  \end{itemize}
\end{frame}

\begin{frame}[fragile=singleslide]{Qemu et Scratchbox}
  \begin{itemize} 
  \item Installation
    \begin{lstlisting}
host% apt-get install scratchbox2
host$ mkdir scratchbox
host$ cd scratchbox
    \end{lstlisting}
  \item Initialisation de l'environement. Nous utilisons qemu-arm pour
    lancer nos binaire cibles et arm-linux-gcc comme compilateur
    \begin{lstlisting} 
host$ sb2-init -c qemu-arm -n A926 arm-linux-gcc
    \end{lstlisting}
  \item Démarrage de l'environement 
    \begin{lstlisting} 
host$ sb2
    \end{lstlisting}
  \end{itemize}
\end{frame}

\begin{frame}[fragile=singleslide]{Qemu et Scratchbox}
  \begin{itemize} 
  \item La cross-compilation est transparente
    \begin{lstlisting} 
qemu$ gcc hello.c -static -o hello-sb2-static
qemu$ file hello-sb2-static
qemu$ ./hello-sb2-static
qemu$ gcc hello.c -o hello-sb2
qemu$ file hello-sb2
    \end{lstlisting}
  \item Comme  sur la cible,  nous avons besoin que  les bibliothèques
    soit bien configurées dans notre environement
    \begin{lstlisting} 
qemu$ cp -a /opt/arm-unknown-linux-uclibcgnueabi/arm-unknown-linux-uclibcgnueabi/sysroot/lib .
qemu$ ./hello-sb2
    \end{lstlisting} 
  \end{itemize} 
\end{frame}





  %
% This document is available under the Creative Commons Attribution-ShareAlike
% License; additional terms may apply. See
%   * http://creativecommons.org/licenses/by-sa/3.0/
%   * http://creativecommons.org/licenses/by-sa/3.0/legalcode
%
% Created: 2012-03-13 00:09:08+01:00
% Main authors:
%     - Jérôme Pouiller <jezz@sysmic.org>
%

\part{Les modules}
\section{Création de modules}

  % insmod/rmmmod
  % modprobe
  % Compiler à l'extérieur du kernel
  % Les Macro
  %   Les paramètres
  %   /sys/**/param
  % Les licences
\begin{frame}[fragile=singleslide]{Les modules noyau}{my\_module}
  \lstinputlisting[language=c,firstline=16]{modules/my_module1.c}
\end{frame}

\begin{frame}[fragile=singleslide]{Quelques macros de base}
  Ces macros  permettent de placer  des informations sur  des symboles
  particulier dans module;
  \begin{itemize} 
  \item Déclare  la fonction  à apeller lors  du chargement  du module
  (traditionnellement module\texttt{\_init})
    \begin{lstlisting} 
module_init
    \end{lstlisting} 
  \item Déclare la fonction à  appeller lors du déchargement du module
  (traditionnellement module\texttt{\_exit})
    \begin{lstlisting} 
module_exit
    \end{lstlisting} 
  \item Déclare un auteur du fichier. Peut apparaitre plusieurs fois.
    \begin{lstlisting}
MODULE_AUTHOR
    \end{lstlisting}
  \end{itemize}
\end{frame}
\begin{frame}[fragile=singleslide]{Quelques macros de base}
  \begin{itemize} 
  \item Description du modules
    \begin{lstlisting}
MODULE_DESCRIPTION
    \end{lstlisting}
  \item Version du module
    \begin{lstlisting}
MODULE_VERSION
    \end{lstlisting}
  \item Rendre un symbole visible par les autres modules.  Il sera alors
    pris en compte dans le calcul des dépendances de symboles.
    \begin{lstlisting}
EXPORT_SYMBOL(symbol)
    \end{lstlisting} 
  \item Idem \cmd{EXPORT\_SYMBOL} mais  ne permet sont utilisation que
    pour les modules GPL
    \begin{lstlisting}
EXPORT_SYMBOL_GPL(symbol)
    \end{lstlisting} 
  \item License. Indispensable
    \begin{lstlisting}
MODULE_LICENSE
    \end{lstlisting}
  \end{itemize}
\end{frame}

\begin{frame}[fragile=singleslide]{Parlons des licenses}
  Le noyau est sous license GPL. Néanmoins, le débat est ouvert sur la
  possibilité  qu'un module  propriétaire puisse  se linker  avec.  Le
  débat n'est pas  tranché. Le noyau laisse la  possibilité à l'auteur
  d'exporter   ses   modules   avec   \verb+EXPORT_SYMBOL+   ou   avec
  \verb+EXPORT_SYMBOL_GPL+.
  \\[2ex]
  Si vous développez un module Propriétaire, vous n'aurez pas accès à
  toute l'API du noyau (environ 90\% seulement).
  \\[2ex]
  Il est néanmoins possible de contourner le problème en utilisant un
  module intermédiaire comme proxy logiciel.
  \\[2ex]
  De plus,  un module  propriétaire ne doit  pas compiler  en statique
  avec  le noyau  (le résultat  est alors  considéré comme  un travail
  dérivé du noyau). Il doit rester sous la forme d'un module.
\end{frame}
\begin{frame}[fragile=singleslide]{Les licences}
  En  remplissant  correctement  la  macro  \verb+MODULE_LICENCE+,  le
  système   de  compilation  du   noyau  empêchera   les  compilations
  illégales.   \verb+MODULE_LICENCE+ peut  prendre  plusieurs valeurs:
  GPL, GPL v2, GPL and additionnal rights, Dual MIT/GPL, Dual BSD/GPL,
  Dual  MPL/GPL.   Toutes les  autre  valeurs  sont considérées  comme
  propriétaire.

  \file{/proc/sys/kernel/tainted} (ou  \cmd{uname -a}) indique  si des
  modules propriétaires sont chargé.
  
  Quels conséquences?
  \begin{itemize} 
  \item Pas de support de la part de la communauté
  \end{itemize} 
\end{frame}

\begin{frame}[fragile=singleslide]{Headers}
  Il  n'est  pas   possible  pour  le  noyau  de   compiler  avec  des
  bibliothèques  extérieur.  En  particulier  avec la  libc. Tous  les
  headers habituels sont donc inaccessible dans le noyau.

  Les headers ``publiques'' sont dans \verb+include/+. On retiendra en
  particulier:       \file{linux/module.h},       \file{linux/init.h},
  \file{linux/kernel.h} indispensables pour compiler un module.

\end{frame} 

\begin{frame}[fragile=singleslide]{Les modules noyau}{my\_module}
  \lstinputlisting[language=c,lastline=14]{modules/my_module1.c}
\end{frame}


\begin{frame}[fragile=singleslide]{Les modules noyau}{my\_module}
  Makefile:
  \begin{lstlisting}
obj-m := my_module.o  
  \end{lstlisting}
  Puis, on appelle:
  \begin{lstlisting}
host$ KDIR=/lib/modules/$(uname -r)/build
host$ make -C $KDIR ARCH=arm SUBDIRS=$(pwd) modules
  \end{lstlisting} % $
  Pour améliorer le processus, on ajoute ces lignes dans le Makefile:
  \begin{lstlisting}
KDIR ?= /lib/modules/$(shell uname -r)/build

default:
        $(MAKE) -C $(KDIR) SUBDIRS=$(shell pwd) modules
  \end{lstlisting}
  et on appelle
  \begin{lstlisting}
host$ make ARCH=arm KDIR=../linux-2.6/usb-a9260 
  \end{lstlisting} % $
\end{frame}

  % Compiler à l'interieur du kernel
  % Fonctionnement de Kconfig
\begin{frame}[fragile=singleslide]{Compilation avec Kmake}
  Makefile:
  \begin{lstlisting}
obj-$(MY_COMPILE_OPTION) := my_module.o  
  \end{lstlisting} % $
  \lstinline+$(MY_COMPILE_OPTION)+ sera remplacé par :
  \begin{itemize}
  \item ø: Non compilé
  \item m: compilé en module
  \item y: compilé en statique
  \end{itemize}
  Kconfig
  \lstinputlisting[langage=]{modules/Kconfig}
  \note[item]{detailler les différentes options de Kconfig}
\end{frame}

\begin{frame}[fragile=singleslide]{Gérer les modules}
  \begin{itemize} 
  \item Avoir des informations sur le module
    \begin{lstlisting}
host$ modinfo my_module.ko
    \end{lstlisting} %$
  \item Charger un module
    \begin{lstlisting}
target% insmod my_module.ko
    \end{lstlisting} %$
  \item Décharger un module
    \begin{lstlisting}
target% rmmod my_module
    \end{lstlisting}%$
  \item Afficher le buffer de log du kernel
    \begin{lstlisting}
target$ dmesg
    \end{lstlisting} %$
  \item Charger/décharger un module correctement installé/indexé
    \begin{lstlisting}
target% modprobe my_module
target% modprobe -r my_module
    \end{lstlisting} %$
  \item Mettre à jour le fichier de dépendance
    \begin{lstlisting} 
target% depmod
    \end{lstlisting} %$
  \end{itemize}
\end{frame}

\begin{frame}[fragile=singleslide]{Paramètres}
  Il est possible de passer des paramètres aux modules:
  \begin{lstlisting}
target$ modinfo my_module.ko 
target% insmod my_module.ko param=2  
  \end{lstlisting} %$ 
  Il est possible de passer le paramètre au boot avec \c{my_module.param=2}
\\[2ex]
  Nous   devons  déclarer   le  paramètre   à  l'aide   de   la  macro
  \c{module_param}.
  \begin{lstlisting}[language=c]
module_param(name, type, perm);
  \end{lstlisting} 
  Prend en paramètres:
  \begin{itemize} 
  \item \cmd{name}: Nom de la variable utilisée comme paramètre
  \item  \cmd{type}:  Type   du  paramètre  (\cmd{bool},  \cmd{charp},
    \cmd{byte},  \cmd{short},   \cmd{ushort},  \cmd{int},  \cmd{uint},
    \cmd{long}, \cmd{ulong}) (remarquez que  ca ne correspond pas tout
    à  fait   aux  types   C.  Particulièrement  pour   \cmd{bool}  et
    \cmd{charp})
  \item        \cmd{perm}:        Permissions        du        fichier
    \file{/sys/module/<module>/parameters/<param>}.    (utiliser   les
    macros de \file{linux/stat.h}). Si 0, le fichier n'apparait pas.
  \end{itemize}
  La paramètre doit évidement être alloué:
  \begin{lstlisting}[language=c]
static int param = 0;
  \end{lstlisting} 
  Il est fortement recommandé de documenter le paramètre
  \begin{lstlisting}[language=c]
MODULE_PARM_DESC(param, "Display this value at loading and unloading");
  \end{lstlisting} 
Référence : \file{linux/moduleparam.h}

\end{frame}

\begin{frame}[fragile=singleslide]{\texttt{/sys}}
  Etudions \file{/sys}
  \begin{itemize}
  \item                       \file{/sys/module/my\_module/parameters}:
    paramètres. Modifiable si déclaré modifiables
  \item \file{/sys/module/my\_module/sections}:  des info sur  la zone
    de chargement
  \end{itemize}
\end{frame}

  %   Le coding style
  %   La documentation
  %   Gestion des erreurs:
  %       -EAGAIN, etc...
  %       goto
  %        errno.h
\subsection{Programmer}
\begin{frame}[fragile=singleslide]{Le Coding Style}
  \begin{itemize} 
  \item On  ne doit pas avoir besoin  d'utiliser l'ascenseur (vertical
    ou horizontal)  pour lire  une fonction (bref:  80 colonne  et pas
    trop de lignes par fonctions)
  \item Indentation à la tabulation
  \item ... et une tabulation fait 8 espaces
  \item Si  l'indentation vous empêche d'écrire  votre fonction, c'est
    que celle-ci possède trop de niveau d'imbrication
  \item Mettre  au propre  les tabulations et  vérifier la  taille des
    lignes:
    \begin{lstlisting} 
host$ scripts/cleanfile my_module.c
    \end{lstlisting} %$
  \item Incolade sur la même ligne que les bloc
  \item ... sauf pour les fonctions
  \item  Pour indenter correctement votre code
    \begin{lstlisting}
host$ apt-get indent
host$ scripts/Lindent my_module.c
    \end{lstlisting} 
  \end{itemize}
\end{frame}

\begin{frame}[fragile=singleslide]{CodingStyle}
  \begin{itemize} 
  \item Les variable locale doivent être courtes
  \item Pas de CamelCase
  \item Si vous n'arriver pas à  vous y retrouver, c'est que vous avez
    trop de variables locales
  \item  Il faut  préserver  la  propreté de  l'espace  de nommage  en
    préfixant  par statique  tout ce  qui  ne doit  pas être  expporté
    (fonctions, variables)
  \item Référence: \file{Documentation/CodingStyle}
  \end{itemize} 
\end{frame} 

 % A couper en deux: une  partie au niveau de "arboresence de sources"
 % et une partie au niveau de l'API
  % Documentation
  % API non stable
  % generer
  %    pages de man
  %    html
  % sur le web
  %    kernel/doc
  %    lxr.linux.no
  % par mail
  %   connaitre avoir l'auteur avec AUTOR
  %   connaitre l'auteur avec git blame
\begin{frame}[fragile=singleslide]{La documentation}
  \begin{itemize}
  \item Le noyau n'est pas un modèle de documentation
  \item En  revanche, le système de  développement facilite énormément
    la recherche d'informations
  \item  Le noyau utilise  un script  perl appellé  \c{kernel-doc}. Ce
    script parcours le code à la recherche de syntaxe du même type que
    doxygen.  (Référence: \file{Documentation/kernel-docs.txt})
  \item  Un  commentaire ne  doit  pas ce  substituer  à  un code  pas
    illisible.  Préférez   réécrire  le  code   plutôt  qu'ajouter  un
    commentaire.
  \item Les fonction exportée doivent être préfixée par un commentaire
    la   décrivant.    Ce  commentaire   ne   doit  surtout   explique
    \emph{comment}  la  fonction marche.   Le  commentaire indique  le
    \emph{quoi},  mais  surtout   le  \emph{pourquoi}  cette  fonction
    existe.
  \item Ces  commentaires sont  placée avant la  \emph{définition} des
    fonctions.  Les  fonction du noyau  sont souvent définie  dans les
    headers.  Du  coup, les commentaires sont parfois  dans le fichier
    \file{.c} et parfois dans le fichier \file{.h}
  \item  Les commentaires  expliquant le  fonctionnement  général d'un
    framework  doivent être  écrit sous  la forme  d'une documentation
    d'architecture et placée dans \file{Documentation/}
  \item Référence: \file{Documentation/CodingStyle}
  \end{itemize}
\end{frame} 

\begin{frame}[fragile=singleslide]{Trouver de la documentation}
  \begin{itemize} 
  \item Dans \file{Documentation} (en utilisant \cmd{grep})
  \item La cible \c{htmldocs}  permet de compiler les documentation au
    format Docbook (\file{Documentation/DocBook})
  \item ... en particulier  \emph{Linux Device Drivers} et \emph{Linux
      Device  Drivers}. Toutefois, ces  compilations sont  loin d'être
    exaustives
  \item  La cible  \c{mandocs}  permet d'extraire  les différente  API
    invoquée dans les DocBook sous forme de pages de man.
  \item    \url{http://kernel.org/doc}     regroupe    les    diverses
    documentations en ligne
  \item Dans les commentaire du noyau (fichiers .c ou .h)
  \item  \url{http://lxr.linux.no} offre  une interface  permettant de
    rapidement  chercher des  symboles  et naviguer  dans  le code  du
    noyau.
  \end{itemize}
\end{frame}

\begin{frame}[fragile=singleslide]{Trouver de l'aide}
Demander de l'aide à la communauté
    \begin{itemize}
    \item Le  fichier MAINTAINERS contient la liste  des mainteneur de
      chaque sous-système
    \item Les entrées commencant par \texttt{L:} indique des mailing
      list
    \item Notons \email{linux-newbie@vger.kernel.org}
    \item MAINTAINERS  liste aussi les sites  officiel, les fichiers
      de documentation, etc...   pour les différents sous-projets du
      noyau
    \item \c{git blame} permet de savoir qui a modifier la fonctions
      ou les  lignes qui nous  intéresse. Cela permet  de s'oreinter
      vers le bon groupe
    \end{itemize}
\end{frame} 

\begin{frame}[fragile=singleslide]{L'API non-stable}
  \begin{itemize} 
  \item  L'API   non-stable  ne   facilite  pas  la   maintenance  des
    documentations
  \item  De même,  la maintenance  des modules  hors du  noyau demande
    beaucoup d'efforts
  \item  C'est  encore pire  de  maintenir  un  module pour  plusieurs
    versions (driver nvidia par exemple)
  \item En  revanche, l'API non  stable permet des  développement plus
    agiles)
  \item  Lors  qu'un développeur  modifie  une  API,  il doit  patcher
    l'ensemble des driver utilisant l'ancienne API.
  \item  On préfère  souvent  laisser l'ancienne  et  la nouvelle  API
    cohabiter en demandant au mainteners de migrer vers la nouvelle
  \item Il n'y  a pas de raisons de maintenir  un driver à l'extérieur
    du noyau, sauf les problèmes de licence
  \item  Il peut être  couteux en  temps d'intégré  un module  dans le
    mainsteam. Les  mainteneurs vous demanderons  surement de corriger
    beaucoup de chose.
  \item Il faut  compter 2 à 4 fois le  temps de développement initial
    pour intégrer un module dans le mainstream
  \item Une fois que votre module  est intégré dans le mainstream et a
    passé  la phase  de  staging, il  sera  théoriquement maintenu  et
    supporté pour toujours.
  \item Référence \c{Documentation/stable_api_nonsense.txt}
  \end{itemize} 
\end{frame} 

\begin{frame}[fragile=singleslide]{Gestion des erreurs}
  \begin{itemize} 
  \item La plupart des fonctions retournant des \c{int} retournent une
    valeur négative en cas d'erreur
  \item Les  valeurs retournée correspondent aux  inverses des erreurs
    Posix  définis dans  \file{errno.h} (ie:  \c{-EAGAIN}). Référence:
    \emph{errno(3)}
  \item  Les  fonctions  renvoyant des  pointeurs  peuvent
    retourner NULL ou des valeurs spéciale en cas d'erreur
  \item Utiliser les macro \c{PTR_ERR}, \c{ERR_PTR} et \c{IS_ERR}
  \item Le segfault n'existe pas  dans le noyau.  Par conséquent, vous
    n'avez pas de garde fou (bon, il  y a bien les Oops, mais ca n'est
    pas garanti)
  \item Référence \file{linux/err.h}
  \item Vous devez systèmatiquement gérer les erreur retourner par les
    sous-fonctions  même  si  il  vous semble  impossible  qu'elle  se
    produise.
  \item Certaines pannes  matérielles peuvent amener des comportements
    qui paraissent  impossibles. Facilité le  travaille des débuggueur
    en détectant ce type d'erreur le plus tôt possible
  \end{itemize}
\end{frame}
\begin{frame}[fragile=singleslide]{Gérer les erreurs}
  \begin{itemize} 
  \item Lorsque vous gérez  une erreur, vous devez \emph{défaire} tout
    ce qui à déjà été fait
  \item Une manière commnue  de gérer les erreur d'initialisation dans
    le noyau est d'utiliser \c{goto}:
    \begin{lstlisting}
ptr = kmalloc(sizeof(device_t));
if (!ptr) {
    ret = -ENOMEM
    goto err_alloc;
}
dev = init(&ptr);
if (dev) {
    ret = -EIO
    goto err_init;
}
return 0;

err_init:
    free(ptr);
err_alloc:
    return ret;
    \end{lstlisting} 
  \end{itemize} 
\end{frame} 



\section{Notre premier device}


\note{Il s'agit ici d'un gros morceau, mais il est traité en qqs slides}
\begin{frame}[fragile=singleslide]{Communiquer avec le noyau}{char\_dev}
  \begin{itemize}
  \item Voie classique pour un driver pour communiquer avec le noyau
  \item Major, Minor
    \begin{lstlisting}
target% mknod my_chardev c <major> <minor>
    \end{lstlisting}
  \item Communication en écrivant/lisant des données
    \begin{lstlisting}
target% insmod my_chardev.ko
target% echo toto > my_chardev
target% cat my_chardev
    \end{lstlisting}
  \item Possibilité de faire un peu de configuration par les ioctl
  \end{itemize}
\end{frame}

\begin{frame}[fragile=singleslide]{Communiquer avec le noyau}{char\_dev}
  Exo: Faire un pipe avec un buffer:
  \begin{lstlisting}
target% insmod my_chardev.ko
target% echo toto > my_dhardev
target% cat my_chardev
toto
  \end{lstlisting}
%% TODO: Présenter les différentes fonction ci-dessous
\note{Motrer le code, décrire le contenu}
  \begin{itemize}
  \item \verb+copy_to_user+, \verb+copy_from_user+
  \item \verb+kcalloc+, \verb+kfree+
  \item \verb+register_chrdev+, \verb+unregister_chrdev_region+
  \item   \verb+mutex_lock+,  \verb+mutex_unlock+,  \verb+mutex_init+,
    \verb+mutex_destroy+
  \item \verb+memmove+
  \end{itemize}
\end{frame}

\begin{frame}[fragile=singleslide]{Communiquer avec le noyau}{char\_dev}
   Vérifions votre comportement:
   \begin{lstlisting}
target% insmod my_chrdev.ko
target% mknod my_chrdev c 251 0
target$ for i in {1..64}; echo "$i " > my_chrdev
target$ cat my_chrdev
target% rmmod my_chrdev
target% insmod my_chrdev.ko buf_size=2
target$ echo foo > /dev/my_chrdev
target$ cat /dev/my_chrdev
target% rmmod my_chrdev
target% insmod my_chrdev.ko buf_size=-1
    \end{lstlisting} % $
\end{frame}
 
% \begin{frame}{Les ioctls}
%   C'est un appel système qui permet de faire passer une structure
%   quelconque à un device.\\
%   Pour appeller un ioctl, il faut un device, le numéro de l'IOCTL et
%   l'arguments.\\
%   Les  numéro d'IOCTL  se  décode ainsi  (attention,  ca n'est  qu'une
%   norme, et elle a ses exeception, principalement powerpc):
% %   \begin{lstlisting}
% %  bits    meaning
% %  31-30  00 - no parameters: uses _IO macro
% %         10 - read: _IOR
% %         01 - write: _IOW
% %         11 - read/write: _IOWR
% % 
% %  29-16  size of arguments
% % 
% %  15-8   ascii character supposedly
% %         unique to each driver
% % 
% %  7-0    function #
% % 
% % 
% % So for example 0x82187201 is a read with arg length of 0x218,
% % character 'r' function 1. Grepping the source reveals this is:
% % Les ioctl doivent être unique par device. Mais on préfère qu'il soit unique
% % sur tout le système.
% % 
% % Les macro _IOR, _IOW, _IORW et _IO nous aident:
% % #define CHANGE_BUF_SZ _IOR(42, 1, int)
% %    \end{lstlisting}

% Aller plus loin: \file{device.h}: Implémentation d'un device complet.
% \end{frame}


  % Programmer
  %   la libc
  %   Big Endian/Little Endian
  %   Preemption (nous y reviendrons)
  %   Les espace mémoire : copy_to_user copy_from_user
  %   Segfault: n'existe pas
  %    Les types: 
  %      u8...u64, s8...s64
  %      __le16..__le64 __be16..__be64
  %      void* et unsigned long et phys_addr_t
 %       linux/types.h
  %      bool mais pas utilisé
  %      Les types de la libc, utilisé pour communique avec l'espace utilisateur: pid_t, uid_t, etc...
  % Les fichier device
  %   mknod
  %   devtmpfs
  % Les appels systèmes
  % Les structure de pointeurs sur fonctions




   

\begin{frame}{Fonctions standards de string.h}
\begin{itemize} 
\item memset, memmove, memcpy, memcmp, memchr, etc...
\item strlen, strcpy, strcmp, strcat, strncat, strchr, etc...
\item  \c{void  *kmemdup(const void  *src,  size_t  len, gfp_t  gfp)},
\item \c{char *kstrdup(const char   *s,   gfp_t   gfp)}   et   \c{char
    *kstrndup(const char *s, size_t max, gfp_t gfp)}
\item sprintf, sscanf
\end{itemize} 
\end{frame} 

\begin{frame}{Les listes}
Implémentation assez singulière de listes génériques en C. 
\begin{lstlisting} 
#define container_of(ptr, type, member) ({ \ 
   const typeof( ((type *)0)->member ) *__mptr = (ptr); \
   (type *)( (char *)__mptr - offsetof(type,member) );})
\end{lstlisting} 
(Inclure linux/list.h)
\begin{itemize}
\item Declarer un type pour les noeuds de sa liste
\item Ajouter un attribut de type \c{struct list_head} à son type

\item Une liste vide est alors composé d'une simple \c{struct list_head} à déclarer ainsi: \c{LIST_HEAD(my_var)}
\item Pour les fonction d'ajout, suppression, etc... tout est géré avec les \c{list_head}
\item Pour accèder au données liées à un \c{list_head}, il faut utiliser la macro \c{} (qui utilise \c{container_of})
\end{itemize} 
\begin{lstlisting}
#include <linux/list.h>
stuct my_node_t {
  struct list_head node;
  /* other attributes */
}
void f() {
   static LIST_HEAD(my_list);
   struct my_node_t new_node; 
   struct my_node_t *i; 
   printf("%d\n", list_empty(my_list));
   list_add(&new_node->node, &my_list);
   printf("%d\n", list_size(my_list));
   list_for_each_entry(i, &my_list, node) {
   }
} 
\end{lstlisting} 
\end{frame} 

\begin{frame}{Barrière mémoire}
  Le CPU peut optimiser du code en réordonnancant les instructions. Ca
  peut être problèmatique dans le cas  des accès IO ou sur des système
  SMP. Utilisation de barrières (Documentation/memory-barriers.txt):
\begin{itemize}
\item rmb() : Pas de réordonnancement des lectures avant et après
\item wmb() : Pas de réordonnancement des écriture avant et après
\item mb() : rmb+wmb
\end{itemize} 
\end{frame} 


\section{Interruptions}

\subsection{wait queue}

\begin{frame}{API wait queue}
  \begin{itemize} 
Declarer:
  \item \c{DECLARE_WAIT_QUEUE_HEAD(my_queue)} : initialisation lors de la déclaration
  \item \c{wait_queue_head_t my_queue; init_waitqueue_head(&my_queue)}: Déclaration et initialisation séparés
Attendre
  \item \c{void wait_event(queue, condition)}: Attend un evenement sur la queue ET que la condition soit vérifiée
  \item \c{int wait_event_killable(queue, condition)}: Idem mais retourne une erreur si executé dans un appel system et le processus est tué avec SIGKILL
  \item \c{int wait_event_interruptible(queue, condition)}: Idem mais retourne une erreur si executé dans un appel system et le processus recoit un signal
  \item \c{void wait_event_timeout(queue, condition)}: Idem wait_event mais avec un timeout
  \item \c{void wait_event_tinterruptible_imeout(queue, condition)}: wait_event_timeout + wait_event_interruptible
Réveiller
\item \c{wait_up(queue)} Réveil tous les processus en attente sur la queue
\item \c{wait+uo_interruptible(queue)} Idem mais, seulement les interruptibles
  \end{itemize} 
\end{frame} 


\subsection{Gestion de la concurrence}

\begin{frame}{Désactivation des interruption et de la préemption}
local_irq_save(flags) : Sauve l'état des interuptions dans flags et désactive les interruptions 
local_irq_disable() : Désactive les interruptions
local_bh_disable();: Désactive les softirq (bottom halves)
preempt_disable(): Désactive la préemption

local_irq_disabled() : Retourne vrai si les interruption du CPU local sont désactivées

local_irq_restore(flags) : Sauve l'état des interuptions dans flags et désactive les interruptions 
local_irq_enable() : Désactive les interruptions
local_bh_enable();: Désactive les softirq (bottom halves)
preempt_enable(): Désactive la préemption
\end{frame} 

\begin{frame}{Spin Lock}
Dans certains cas,  une tâche peut avoir besoin
  d'un accès  exclusif à  une ressource potentiellement  utilisée dans
  une interruption. Dans  ca cas, il est necessaire  de désactiver les
  interruption   lors   de   l'accès   à  la   ressource.   Néanmoins,
  l'interruption  peut se  produire sur  un  autre CPU.  Il est  alors
  nécessaire de se protèger contre cette éventualité
  \c{linux/spinlock.h} 
\begin{itemize} 
\item Il s'agit d'une attente active sur une section critique
\item Ne traite pas les problème de concurence mais de parallèlisme uniquement. Par conséquent, uniquement en environnement SMP
\item Principalement utilisé lors de en complément de la désactivation des interruption
\item La préhemption est aussi désactivé durant un spin_lock
\end{itemize} 
\begin{itemize}
\item \c{DEFINE_SPINLOCK(lock)},  \c{spin_lock_init(spinlock_t *lock)} Initialise un spin lock
\item \c{spin_lock(spinlock_t *lock)}, \c{spin_trylock(spinlock_t *lock)}, \c{spin_unlock(spinlock_t *lock)} Idem que les fonction \c{mutex_*}
\item \c{spin_lock_irqsave(spinlock_t *lock)}, \c{spin_unlock_irqrestore(spinlock_t *lock)} Spinlock et désactive/réactive les interruptions sur le processeur courant. Les flags contiennent l'états du mask d'interruption
\item \c{spin_lock_bh(spinlock_t *lock}, \c{spin_unlock_bh(spinlock_t *lock)} spinlock et désactive/réactive les softirq.
\end{itemize} 
\end{frame} 

Faire un tableau
\begin{frame}{Alternatives}
  Opérations atomiques
  \begin{itemize}
  \item linux/atomic.h Documentation/atomic_ops.txt
  \item static atomic_t my_counter = ATOMIC_INIT(i) Initialisation
  \item void atomic_set(atomic_t *v, int i)
  \item int atomic_read(atomic_t *v)
        void atomic_inc(atomic_t *v)
        void atomic_dec(atomic_t *v)
        void atomic_add(int i, atomic_t *v)
        void atomic_sub(int i, atomic_t *v)
        int atomic_inc_return(atomic_t *v)
        int atomic_dec_return(atomic_t *v)
        int atomic_add_return(int i, atomic_t *v)
        int atomic_sub_return(int i, atomic_t *v)
        int atomic_inc_and_test(atomic_t *v)
        int atomic_dec_and_test(atomic_t *v)
        int atomic_xchg(atomic_t *v, int new)
        int atomic_cmpxchg(atomic_t *v, int old, int new)


        void set_bit(unsigned long nr, volatile unsigned long *addr)
        void clear_bit(unsigned long nr, volatile unsigned long *addr)
        void change_bit(unsigned long nr, volatile unsigned long *addr)
        int test_bit(unsigned long nr, volatile unsigned long *addr)
        int test_and_set_bit(unsigned long nr, volatile unsigned long *addr)
        int test_and_clear_bit(unsigned long nr, volatile unsigned long *addr)
        int test_and_change_bit(unsigned long nr, volatile unsigned long *addr)
        long cas(long *mem, long old, long new) (Compare and Swap)

  \item 
  \end{itemize} 
local_t (Documentation/local_ops.txt) Similaire aux atomic_t mais locaux à un CPU

 Read-Copy-Update (RCU)
\end{frame} 


\begin{frame}{ Debugguer}
Il est possible d'utiliser directemenr printk en specifiant le degré d'importance de l'information:
printk(KERN_ALERT)
Néanmoins, il est maintenant conseillé d'utiliser les macro pr_cont , pr_debug, pr_info, pr_notce, pr_warning, pr_err, pr_crit, pr_alert, pr_emerg

Les pr_debug ne sont compilé que si l'une des option CONFIG_DEBUG ou CONFIG_DYNAMIC_DEBUG est active. Dans le cas de DYNAMIC_DEBUG, il est posisble d'activer les messages namiquement en passant par debugfs (cf Documentation/dynamic-debug-howto.txt)
(mount -t debugfs none /sys/kernel/debug si nécessaire)

BUG()
BUG_ON(condition)
\end{frame} 

  %
% This document is available under the Creative Commons Attribution-ShareAlike
% License; additional terms may apply. See
%   * http://creativecommons.org/licenses/by-sa/3.0/
%   * http://creativecommons.org/licenses/by-sa/3.0/legalcode
%
% Created: 2012-03-13 00:09:31+01:00
% Main authors:
%     - Jérôme Pouiller <jezz@sysmic.org>
%

\part{Debuguer}

\begin{frame}
  \partpage
\end{frame}

\begin{frame}
  \tableofcontents[currentpart]
\end{frame}

\section{Afficher des informations}

\begin{frame}[fragile=singleslide]{\texttt{printk}}
  \begin{itemize} 
  \item \c{printk} affiche les information sur la console et les copie
    dans  un  buffer.   Il  est  possible d'afficher  ce  buffer  avec
    \cmd{dmesg}
  \item   Il  est  possible   d'utiliser  directemenr   \c{printk}  en
    specifiant le degré d'importance de l'information:
    \begin{lstlisting} 
printk(KERN_ALERT "Aie\n");
    \end{lstlisting} 
  \item Notez l'absence de virgule entre \c{KERN_ALERT} et la chaîne
  \item Néanmoins,  il est maintenant conseillé  d'utiliser les macros
    \c{pr_cont}, \c{pr_devel}\c{pr_debug}, \c{pr_info}, \c{pr_notice},
    \c{pr_warning},     \c{pr_err},     \c{pr_crit},     \c{pr_alert},
    \c{pr_emerg}
  \item Existe aussi  suffixé de \c{_once} pour n'être  affiché que la
    première fois.
  \item  \c{print_hex_dump_bytes} et  \c{print_hex_dump}  affichent un
    dump en hexadécimal d'un buffer.  La seconde est la forme générale
    qui  permet de  faire à  peu près  la même  chose que  la commande
    \c{hexdump}
  \end{itemize}
\end{frame}

\begin{frame}[fragile=singleslide]{\texttt{printk}}
  \begin{itemize} 
  \item  \c{printk} est  compatible avec  la norme  C99 et  ajoute des
   extensions  pour  pretty-printer  certains pointeurs  (\c{\%p}).  En
   particulier:
     \begin{itemize}
    \item \c{\%pF}  pour résoudre le  nom d'une fonction du  segment de
      text
    \item \c{\%pUB} et \c{\%pUL} pour afficher les UUID ou les GUID
    \item \c{\%pM} pour afficher des adresses MAC
    \item \c{\%pI4} pour affichier des adresses IPv4
    \item \c{\%pI6} et \c{\%pI6c} pour afficher des adresses IPv6
    \item Les  formats d'IP peuvent être suivit  de h, n, l  ou b pour
      demander des conversions d'endian avant d'afficher
    \item \c{\%m} pour \c{errno}
    \end{itemize} 
  \item                Référence                \file{linux/printk.h},
    \file{Documentation/printk-formats.txt}
  \item Il  est aussi possible de  provoquer des dump de  la pile avec
    les macro \c{BUG()}, \c{BUG_ON(condition)} et \c{BUG_ON_NULL(ptr)}
  \item  Les \c{pr_debug}  ne sont  compilé  que si  l'une des  option
    \c{CONFIG_DEBUG} ou \c{CONFIG_DYNAMIC_DEBUG} est active
  \item Dans  le cas de  \c{DYNAMIC_DEBUG}, il est  possible d'activer
    les messages atomiquement en passant par debugfs
  \item Référence: \file{Documentation/dynamic-debug-howto.txt}
  \end{itemize} 
\end{frame} 

\subsection{Dynamic printk}

\begin{frame}[fragile=singleslide]{debugfs}
  \begin{itemize} 
  \item Un certain nom d'outils de debug sont commandé par debugfs
  \item Il s'agit d'un type de partition virtuel
  \item Activé avec 
    \begin{lstlisting} 
mount -t debugfs none /sys/kernel/debug
    \end{lstlisting} 
  \item Il est possible d'obtenir le statut de dprintk avec
    \begin{lstlisting} 
 cat /sys/kernel/debug/dynamic_debug/control
    \end{lstlisting} 
  \item ... et d'activer de nouvelles traces avec 
    \begin{lstlisting} 
echo 'file hello.c +plf' >  /sys/kernel/debug/dynamic_debug/control
    \end{lstlisting} 
  \end{itemize} 
\end{frame} 

%% TODO parler ici des sysrq. Reprendre le docbook de kgdb et de freeelectron
%\begin{frame}[fragile=singleslide]{SysRq}
%\end{frame}

\section{kgdb}

  % kgdb
\begin{frame}[fragile=singleslide]{Debuguer le noyau}
  Activation d \cmd{kgdb} permettant d'embarquer \cmd{gdbserver} dans
  le noyau:
  \begin{itemize}
  \item Compilation
    \begin{lstlisting}
host$ cp -a build build-dbg
host$ make O=build-dbg ARCH=arm CROSS_COMPILE=arm-linux- menuconfig
... Compile the kernel with debug info ...
... KGDB: kernel debugging with remote gdb ...
host$ make -j3 O=build-dbg ARCH=arm CROSS_COMPILE=arm-linux- uImage
    \end{lstlisting} 
  \item L'image ELF est beaucoup plus grosse
    \begin{lstlisting}
host$ ls -l build-dbg/vmlinux build/vmlinux
host$ cp build-dbg/uImage /srv/tftp/uImage-2.6.33.7-dbg
    \end{lstlisting} 
  \end{itemize}
\end{frame}
\begin{frame}[fragile=singleslide]{Debuguer le noyau}
  \begin{itemize}
  \item  Passage des arguments  \cmd{kgdboc} indiquant  quel interface
    \cmd{kgdb} doit utilisé et  \cmd{kgdbwait} indiquant que kgdb doit
    attendre notre connexion avant de continuer le boot
    \begin{lstlisting}
uboot> set bootargs ${bootargs} kgdboc=ttyS0 kgdbwait
uboot> tftp
<Ctrl-a q>
host$ arm-linux-gdb vmlinux 
gdb> target remote /dev/ttyUSB1
    \end{lstlisting} 
    \item Référence: \file{htmldocs/kgdb.html}
  \end{itemize}
\end{frame}

% \begin{frame}[fragile=singleslide]{Debuguer le noyau}
%   Nécessite  une modification  du driver  \file{atmel\_serial}  afin de
%   permettre   l'utilisation  de   son  port   serie   par  \cmd{kgdb}.
%   Modification  apportée par  Albin Tonerre  backportée de  la version
%   2.6.34 du noyau (nous y reviendrons):
%   \begin{lstlisting}
% host$ patch -p1 < 0001-serial-atmel_serial-add-poll_get_char-and-poll_put_c.patch 
%   \end{lstlisting} 

% \note[item]{testé avec la version 2.6.37, mais devait fonctionner comme ca}
% \end{frame}

\begin{frame}[fragile=singleslide]{Commandes gdb utiles}
  Gestion de l'exécution
  \begin{itemize}
  \item Point d'arrêt à l'adresse \verb+0x9876+
    \begin{lstlisting}
gdb> b *0x9876
    \end{lstlisting}
  \item \verb+b  hello.c:30+ Point  d'arrêt à la  ligne 30  du fichier
    \file{hello.c}
    \begin{lstlisting}
gdb> b hello.c:30
    \end{lstlisting}
  \item Point d'arrêt sur la fonction \cmd{main}
    \begin{lstlisting}
gdb> b main
    \end{lstlisting}
  \item Continuer un programme arrêté
    \begin{lstlisting}
gdb> c
    \end{lstlisting}
  \item Démarrer le programme avec \verb+arg+ comme argument
    \begin{lstlisting}
gdb> r arg
    \end{lstlisting}
  \end{itemize}
\end{frame}
  
\begin{frame}[fragile=singleslide]{Commandes gdb utiles}
  Obtenir des informations
  \begin{itemize}
  \item Voir la pile d'appel
    \begin{lstlisting}
gdb> bt
    \end{lstlisting}
  \item Afficher les variable locales à la fonction
    \begin{lstlisting}
gdb> i locales
    \end{lstlisting}
  \item Afficher les arguments de la fonction
    \begin{lstlisting}
gdb> i args
    \end{lstlisting}
  \item Afficher la valeur de \verb+a+
    \begin{lstlisting}
gdb> p a
    \end{lstlisting}
  \item Afficher la valeur de \verb+abs(2 * (a - 4))+
    \begin{lstlisting}
gdb> p abs(2 * (a -  4))
    \end{lstlisting}
  \end{itemize}
\end{frame}

\begin{frame}[fragile=singleslide]{Commandes utiles}
  Debuguer à chaud
  \begin{itemize}
  \item Attacher \verb+gdb+ à un processus existant
    \begin{lstlisting}
gdb> attach 1234
    \end{lstlisting}
  \item Arrêter le debug sans arrêter le processus
    \begin{lstlisting}
gdb> detach
     \end{lstlisting}
   \end{itemize}
\end{frame}

\begin{frame}[fragile=singleslide]{Commandes utiles}
  Et plus...
  \begin{itemize}
  \item Afficher  la valeur de \verb+i+  à chaque fois  que l'on passe
    sur le point d'arrêt 1
    \begin{lstlisting}
gdb> command 1
> silent
> p i
> c
> end
    \end{lstlisting}
  \item Arrêter le programme lorsque la variable \verb+i+ est modifiée
    ou lue
    \begin{lstlisting}
gdb> watch i
gdb> awatch i
      \end{lstlisting}
    \item Passer d'une thread à une autre
      \begin{lstlisting}
gdb> thread 2
gdb> thread 1
      \end{lstlisting}
    \end{itemize}
\end{frame}

\begin{frame}[fragile=singleslide]{Kgdb et les modules}
  \begin{itemize} 
  \item Les modules sont chargé dynamiquement. Ils ne sont pas contenu
    dans le fichier \file{vmlinux}
  \item Il  est nécessaire d'indiquer à  gdb de charger  le module que
    l'on souhaite débuguer
  \item Néanmoins, gdb ne peut pas savoir à quel addresse à été chargé
    le module
  \item    Ces    informations    peuvent   être    récupérées    dans
    \file{/sys/modules/*/sections}
  \item On peut alors les donner à gdb avec \c{add-symbol-file}:
    \begin{lstlisting} 
target% cat /sys/modules/my_modules/sections
host$ gdb vmlinux
> target remote 192.168.1.55:2345
> add-symbol-file my_modules.ko 0xd0832000 \
     -s .bss 0xd0837100 -s .data 0xd0836be0
    \end{lstlisting} 
  \item    Il   existe   des    scripts   effectuant    la   procédure
    automatiquement. A adapter suivant vos besoins.
  \end{itemize}
\end{frame} 

\section{Tracker la mémoire}

\begin{frame}[fragile=singleslide]{kmemleak}
  \begin{itemize} 
  \item Active  un algorithme proche  d'un garbage collector.  Mais au
    lieu  de  libérer  la  mémoire, enregistre  simplement  les  blocs
    perdus.
  \item Le résultat se trouve dans \file{/sys/kernel/debug/kmemleak}
  \item La procédure classique:
    \begin{lstlisting} 
target% echo clear > /sys/kernel/debug/kmemleak
target% modprobe my_module
...
target% modprobe -r my_module
target% echo scan > /sys/kernel/debug/kmemleak
target% cat /sys/kernel/debug/kmemleak
    \end{lstlisting} 
  \item Référence: \file{Documentation/kmemleak.txt}
  \end{itemize} 
\end{frame}

\begin{frame}[fragile=singleslide]{kmemcheck}
  Modifie la  MMU de manière à  ce qu'une exception  soit déclenchée à
  chaque accès à la mémoire. Lors qu'un accès en lecture est effectué,
  kmemcheck  vérifie si  l'octet à  déjà été  écrit auparavent.  Si ca
  n'est pas le cas, une erreur du même type qu'un Oops est générée
  \\[2ex]
  (Pas disponible sur ARM)
  \\[2ex]
  Référence: \file{Documentation/kmemcheck.txt}
\end{frame} 

\section{Outils à base de Kprobe et apparentés}

\subsection{kprobe}

\begin{frame}[fragile=singleslide]{Kprobe}
  D'un point  de vue générique, Les Kprobes  permettent de d'effectuer
  une  action  lors  du  passage  de l'exécution  d'une  case  mémoire
  particulière (Référence: \file{Documentation/kprobes.txt})
  \\[2ex]
  Dans la partie userspace, il  existe une api d'avoir des informations
  statistique sur le passage sur  une ligne.  Il suffit d'indiquer une
  adresse (ou un symbole) à prober :
  \begin{lstlisting} 
echo 'p do_fork' > /sys/kernel/debug/tracing/kprobe_events 
  \end{lstlisting} 
  ... et de l'activer
  \begin{lstlisting} 
echo 1 > /sys/kernel/debug/tracing/events/kprobes/p_do_fork_0/enable             
  \end{lstlisting} 
  On peut maintenant lire les résultats dans:
  \begin{lstlisting} 
cat /sys/kernel/debug/tracing/trace
  \end{lstlisting} 
  Référence: \file{Documentation/trace/kprobetrace.txt}
\end{frame} 

\subsection{Tracepoint}

\begin{frame}[fragile=singleslide]{Tracepoint}
  Les tracepoints sont  des endroit spécifié dans le  code du noyau qui
  peuvent sembler intéressant à mesurer:
  \begin{itemize}  
  \item Pas de possibilité de les déclarer n'importe ou dans le code
  \item Permet d'obtenir plus d'informations que les kprobes
  \item Les développeur du noyau  ont estimé que l'endroit était pour y mettre un tracepoint et ont configuré le format adéquate (cf.
    \file{/sys/kernel/debug/tracing/events/*/format})
  \item  L'overhead pour  un  tracepoint désactivé  est  un poil  plus
    important que  pour les kprobe  vu qu'il n'y  a pas de patch  à la
    volée du code (mais du coup l'usage est permit en XIP)
  \end{itemize} 
  Pour en obtenir la liste des tracepoints:
  \begin{lstlisting} 
find /sys/kernel/debug/tracing/events/ -type d
  \end{lstlisting} 
  Il  est évidement  possible  de de  définir  ses propres  tracepoints
  (cf. \file{Documentaiton/trace/tracepoints.txt})
  \\[2ex]
  Référence: \file{Documentation/trace/events.txt}
\end{frame} 

\subsection{Ftrace}

\begin{frame}[fragile=singleslide]{ftrace}
  Permet  d'instrumenter certaines  latence  particulière. Il  utilise
  l'option  \c{-pg} de  \c{gcc}. Cela  ajoute un  appel  à \c{_mcount}
  avant chaque  appel de fonction.  Contrairement à  kprobe, ftrace ne
  peut se placer qu'au début d'une fonction.
  \\[2ex]
  Néanmoins, ftrace est très utilisé dans le temps réel, la mesure des
  temps de latence et l'étude du scheduling.
  \\[2ex]
  Reference:                      \file{Documentation/trace/ftrace.txt}
  \file{Documentation/trace/ftrace-design.txt}
\end{frame} 

\subsection{Oprofile}

\begin{frame}[fragile=singleslide]{oprofile}
  Les CPU  modernes possèdent des  registres de debugs  et breakpoints
  hardware capable de profiler  le code. Oprofile permet de paramétrer
  et  de  visualiser le  contenu  de  ces  registres.  Oprofile  n'est
  disponible que sur certaines architectures.
\end{frame} 

\subsection{Perf}

\begin{frame}[fragile=singleslide]{perf}
  perf vous offre une interface pour accéder :
  \begin{itemize} 
  \item  aux tracepoint
  \item  aux kprobe
  \item  aux registres de debug hardware et au breakpoints hardware
  \end{itemize} 
  Ainsi,  perf offre  une  interface plus  complète  que oprofile  (et
  disponibles sur plus d'architecture)
  \\[2ex]
  (perf ne compile pas sur notre carte de test  car notre toolchain est un peu trop
  vieille pour le noyau 3.3 (et j'ai la flemme de la recompiler))
\end{frame} 

\begin{frame}[fragile=singleslide]{SystemTap}
  Permet    d'accéder   à    quasiment   toutes    les   technologies
  précédentes.  Compiler  en  natif,   ce  qui  permet  d'être  très
  rapide... Nécessite une formation à lui-seul.
  \\[2ex]
  Le principe:
  \begin{itemize} 
  \item  On écrit du pseudo-code C
  \item  Il est converti en un modules C
  \item  Il est compilé, puis chargé sur le noyau de la cible
  \item  Le résultats est récupéré par un service appelé \emph{relayfs}
  \end{itemize} 
  Référence: systemtap-doc
  \\[2ex]
  Pour   une   vue  globale   est   différents   outils  d'instrumentation:
  \file{Documentation/trace/tracepoint-analysis.txt}
\end{frame} 



  % kdb
  % Les magic keys
  % Le soption de debug de HACKING
  % Les autres: systemtap, lttng , ftrace, kernel_probe




%  %
% This document is available under the Creative Commons Attribution-ShareAlike
% License; additional terms may apply. See
%   * http://creativecommons.org/licenses/by-sa/3.0/
%   * http://creativecommons.org/licenses/by-sa/3.0/legalcode
%
% Created: 2012-03-13 00:11:43+01:00
% Main authors:
%     - Jérôme Pouiller <jezz@sysmic.org>
%

\part{Contribuer}

  % Créer une branche
  % Créer un patch
  % Créer une serie de patch
  % Appliquer un patch
  % Formater un mail
  % Trouver la mailing list
  % Retravailler une branche

% MLs
% patchwork
% git clone
% git pull
% git branch
% git prepare_patch
% git bisect
% fir rebase



\end{document}

%%% Local Variables: 
%%% mode: latex
%%% TeX-master: t
%%% End: 
