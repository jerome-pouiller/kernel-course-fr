\subsection{Linux}

\begin{frame}{Qu'est-ce que Linux?}
  \begin{itemize}
  \item  Linux ne désigne  que le  noyau
    \note[item]{nous  verrons ce  qu'est le noyau}
  \item  Linux est  souvent  associé aux  outils  GNU d'où  le nom  de
    GNU/Linux
  \item  Systèmes  avec  les  outils  GNU  mais  un  noyau  différent:
    GNU/Hurd, Solaris, etc...
  \item Systèmes Linux sans GNU: Android\note[item]{Bionic}
  \item Le nombre de systèmes  Linux installés est difficile à évaluer
    (en partie car des système Linux embarqués)
  \end{itemize}
\end{frame}


\subsection{L'embarqué}

\begin{frame}{Qu'est-ce que l'embarqué?}
  D'après Wikipedia:
  \begin{quote}
    \justify  Un système embarqué  peut être  défini comme  un système
    électronique et  informatique autonome, qui est dédié  à une tâche
    bien  précise.   Ses   ressources  disponibles  sont  généralement
    limitées.   Cette  limitation  est  généralement  d'ordre  spatial
    (taille limitée) et énergétique (consommation restreinte).
    \\[2ex]
    Les systèmes  embarqués font très souvent  appel à l'informatique,
    et notamment aux systèmes temps réel.
    \\[2ex]
    Le terme de système embarqué désigne aussi bien le matériel que le
    logiciel utilisé.
  \end{quote}
\end{frame}

\begin{frame}{Cible et Hôte}
  \begin{itemize}
  \item  Nous parlerons  de système  cible  (target, la  board) et  de
    système hôte (host, votre machine)
  \item Le  host va nous permettre de  programmer, debugger, contrôler
    le système cible durant la période de développement
  \item Le système cible sera ensuite autonome.
  \item Nous utilisons un Linux sur les deux système. Ca n'est pas une
    obligation (même, si cela facilite certains automatisme).
  \end{itemize}
\end{frame}

\begin{frame}{Qu'est-ce qu'une distribution?}
  \begin{itemize}
  \item Debian, Ubuntu, Meego,  Red Hat, Suse, ... \note[item] {Parler
      de Meego, dériv sur Intel, Nokia, Renault, etc... Dire que c'est
      en plein essort}
  \item Compilations de programmes disponibles pour GNU/Linux
  \item Ensemble de normes et de procédure
  \item Permet de garantir le fonctionnement des programmes distribué
  \item Notre distribution ``Hôte'': Ubuntu % à vérifier
  \item Notre distribution ``Cible'': Aucune
  \end{itemize}
\end{frame}

  \begin{frame}{Présentation du noyau}
    Reprendre la présentaion de eLinux
  \end{frame}
  \begin{frame}{Numerotation du noyau}
    2.4, 2.6, 3.0  
  \end{frame}

  
\subsection{Kmake}

